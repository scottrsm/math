\documentclass{article}
\usepackage{amsmath,amscd}

\title{Why Lebesgue Integration}
\author{R. Scott McIntire}
\date{Nov 1, 2024}
\setlength{\textwidth}{6.5in}
\setlength{\textheight}{7.5in}
\setlength{\hoffset}{-0.75in}
\setlength{\voffset}{-0.75in}

\setlength{\parindent}{0.0in}
\setlength{\parskip}{0.5\baselineskip}


\begin{document}
\maketitle


\section{Better Riemann Integration Convergence Theorems}
Suppose the sequence of functions, $f_n$, converge point-wise to a function $f$.
We wish to find minimal conditions to place on $f_n$ so that the following occurs:

{\bf Theorem?}
\begin{eqnarray}
	\lim_{n \rightarrow \infty}\limits \int f_n(x) \, dx & = & \int \lim_{n \rightarrow \infty}\limits f_n(x) \, dx \label{theorem}
\end{eqnarray}

Like any good technology this simplifies the work of mathematicians 
and applied scientists. For instance, suppose the $f_n$ converges to the function $f$ 
which is the zero function. Also suppose that it is difficult or impossible to 
find an analytic expression for some, most, or all of the integrals of $f_n$. 
An improved convergence theorem could help us to conclude that the limit of the integrals of 
the functions is just $0$ without ever having to attempt the integrations; or, 
derive upper bounds on the integrals to show that 
the sequence of integrals goes to $0$.

However, a typical condition is that the $f_n$ converge 
uniformly to $f$ on a close bounded interval. 
But this rarely happens in many real world examples and you can imagine that
mathematicians back in the day
were getting complaints about such restrictive 
conditions. The people who want a better convergence theorem say that many 
times they don't have this strong condition and yet $(\ref{theorem})$
holds. Unfortunately, other times it does not. So it seems that there are 
conditions less rigid than uniform convergence of the sequence of 
functions which allow $(\ref{theorem})$ to hold.

There are better theorems and they come from a different kind of integration called 
the Lebesgue ({\em measure\/}) Theory of Integration.

Let's stick with Riemann integrals and see if we can find better convergence 
criterion.

\section{Looking for better Convergence Criterion}
Examining how Riemann integration works, we see that we need to bound the upper 
and lower Darboux sums for a given partition of the domain of a function. 
Let's look for an example where things fail; that is, where 
a sequence of functions leads to a function for which we ``lose control'' 
constraining the upper and lower bounds of the sub-intervals in a way that the 
convergence theorem fails. Then we might see what we need to add/impose 
-- in terms of conditions -- on the sequence to make theorem 
$(\ref{theorem})$ true.

Since we are using Riemann functions, 
let's start with discontinuous functions 
and see what can go wrong -- this should be easier. This will also free us from 
trying to make complicated continuous functions that behave badly in the limit.

We restrict ourselves to functions in the domain $[0, 1]$.
Consider the following sequence of functions: 

\begin{eqnarray}
	f_n(x) & = & \begin{cases}  1 \quad \text{if $x$ can be expressed as the fraction, $\frac{p}{q}$ for some $p, q \in {\cal N}$ with $p, q \le n$ ;} \\ 
	  			        		0 \quad \text{otherwise.} 
			   	 \end{cases}
\end{eqnarray}

You can see that the sequence, $f_n$, converges point-wise to the function 

\begin{eqnarray}
	f(x) & = & \begin{cases} 1 \quad \text{if $x$ is rational;} \\ 
			                 0 \quad \text{otherwise.} 
			   \end{cases}
\end{eqnarray}

What can we say about the limit of the integrals of $f_n$?

Well, each $f_n$ is non-zero at a finite number of points so its integral is $0$.
Therefore the limit of integrals is $0$.

What about the integral of $f$? We will show that this integral is bounded above
and below by $0$; meaning that the integral is $0$. The below part is obvious. 
For the upper bound one usually creates a sequence of functions which bound $f$ 
from above and yet as you move out in the sequence their integrals go to $0$.
We will do something a little different. We instead  
construct a sequence of functions $g_n$ which {\em represent\/} $f$ 
in increasing refined ways whose integrals are more amenable to estimation from above.

First, let $\{r_i\}$ be a 1-1 mapping of the positive integers to the rationals.
We know we can do this as the rationals are countable. 

Consider the sets, $\{E_{i,n}\}$, defined by:

$E_{i,n} = \left[r_i - \frac{1}{n 2^{i}}, r_i + \frac{1}{n 2^{i}}\right]$

A given rational number lies in potentially many of the $E_{i,n}$.

Define the functions, $S_i$ by
\begin{eqnarray}
	S_{i}(x) & = & \begin{cases} 1 \quad \text{if $x = r_i$}; \\ 
	 							  0 \quad \text{otherwise.} 
					\end{cases}
\end{eqnarray}

Below we make use of the notation: 

\begin{eqnarray}
I_X(x) & = & \begin{cases} 1 \quad \text{if $x \in X$}; \\ 0 \quad \text{otherwise.} \end{cases}
\end{eqnarray}

Finally, define, $g_n$ by

\begin{eqnarray}
	g_{n}(x) & = & \sum_{i=1}^\infty\limits S_{i}(x) I_{E_{i,n}}(x)
\end{eqnarray}

It is easy to see that each $g_n$ is a {\em representation\/} of the function $f$; 
that is, $f \equiv g_n$.

Note the following facts:
\begin{eqnarray*}
	\int_0^1 S_{i}(x) I_{E_{i,n}}(x)\, dx \le \overbrace{1}^{\text{Max Height}} * \overbrace{2\frac{1}{n2^{i}}}^{\text{Width of $E_{i,n}$}} 
														  & = & \frac{1}{n2^{i-1}} \\
  \sum_{i=1}^\infty \int_0^1 S_{i}(x) I_{E_{i,n}}(x)\, dx & \le & \sum_{i=1}^\infty\limits \frac{1}{n2^{i-1}}  \\ 
														  & =  & \frac{2}{n} \sum_{i=1}^\infty\limits \frac{1}{2^{i}} \\
													  	  & =  & \frac{2}{n}
 \end{eqnarray*}

What does this say about the integral of $f$? 

Therefore, given an $n \in {\cal N}^+$ for any partition of the interval, $[0,1]$, we can bound the integral of $f$ 
(assuming it is Riemann integrable) by
\begin{eqnarray*}
	\int_0^1 f(x) \, dx & =   & \int_0^1 g_{n}(x) \, dx \\
						& \le & \text{(Darboux Upper Sum)}(g_n)  \\
						& \le & \sum_{i=1}^\infty 1 * \text{length}(I_{E_{i,n}}(x)) \\
						& \le & \frac{2}{n}
\end{eqnarray*}

Since this is true for all $n \in {\cal N}^+$, the integral of $f$ must be $\le 0$. 
But, since $f$ is non-negative, its integral must be $\ge 0$. Therefore, the 
integral of $f$ must be $0$.

So, in trying to find an example where the theorem failed, we didn't succeed. 
Even using a sequence of discontinuous functions we still got convergence without 
requiring uniform convergence. This is suggestive that we may not, in general, 
need a very stringent condition for the theorem to hold.

But wait, we are assuming that the function $f$ is a Riemann integrable function. 
Is it?

It turns out that the answer is NO! 

If you look at any given partition of the domain, the approximating lower 
and upper Darboux sums, are always 
always $0$ and $1$ respectively. To see this, note that 
over any sub-interval of the sums the minimum (infimum) value is always 0 and 
the maximum (supremum) is always 1 -- because on any interval there is always a 
rational and an irrational number.

Yet the above analysis suggests that its ``integral'' should exist and its value 
should be $0$.

A contrarian response to the above argument might be:
``What I really want is to have a theorem for a sequence of 
'nice functions'. I don't care that you have found Riemann integrable functions that
don't converge to a Riemann Integrable function''. 

It turns out; however, that there are sequences of nice functions that converge 
to a non-Riemann integrable function as well!

So, in addition to the conditions we need to ensure that we can interchange 
limits with integrals, we also need conditions to ensure that the 
resulting limiting function is integrable.

This situation is similar to one that you would find yourself in if there were 
only rational numbers in your worldview. Let's say that at some point 
you discover an iterative 
algorithm to solve a problem for which there is no easy analytic formula. 
Newton's method is such an algorithm. Imagine coming up with a sequence of {\em numbers\/}
based on Newton's method to find the $\sqrt{x}$. How would you formalize what 
it means for your sequence to converge -- particularly when $x = 2$; that is, when the
square root of $2$ doesn't exist in your world?

We are in an analogous situation. It suggests that 
we try a different approach to integration before we even
begin to think of better convergence theorems. 

However, whatever we come up with, we should get the same value for the 
integrals of ``nice'' functions. In particular, all of the previous integration 
formulas should hold. For instance, it should still be true that: 
$\int_0^{\pi / 2} \sin(x) \, dx = -\cos(x)|_{0}^{\pi / 2} = 0 - (-1) = 1$
In other words, we don't want to lose the connection between differential and
integral calculus.


\section{The Beginnings of a Lebesgue Theory of Integration}

Riemann integration was all about chopping up the {\em Domain\/} of a function 
into a partition of finite intervals and then showing that the difference 
between the upper and lower Darboux sums of the partition 
go to zero as the partition is refined. Basically, approximating the function 
with step functions.

Let's go the other way, based on the example from the previous section, rather 
than step functions, let's break up the function in terms of its {\em Range\/}. 
In our case there are only two ``Range'' values: $0$ and $1$. From this we get an 
analog of Riemann's step functions -- 
so called characteristic functions of sets:

$\chi_{S} = \begin{cases} 1 \quad \text{if $x \in  S$;} \\ 0 \quad \text{otherwise.} \end{cases}$

Although approximating the function, $f$, from the last section using step 
functions is unpleasant, it is very easy using characteristic functions -- the complexity 
is buried in the underlying set, $S$. In our case, we can represent our function {\em exactly\/}
using a linear combination of characteristic functions: $f(x) = \chi_{IR}(x)$.

Here, $IR$ is the set of irrationals on interval, $[0, 1]$.

To find the ``area'' under the curve ``$f(x)$'' we would like to multiply the 
height by the ``width'' of each of the sets. What might this mean?
Well we can think of {\em squeezing\/} all of the points of each set so that there are 
no more holes and then taking the length of the resulting interval.
But how do we make sense of this?

One way to measure the length of these complicated sets is to try different ``coverings'' 
of a set by collections of intervals and relate the ``length'' of a set to the lengths of
the covering intervals. Specifically, as
we know how to measure the length of an interval%
\footnote{We denote the length of an interval, $I$, by $|I|$.}, an upper bound on 
the ``length'' of an arbitrary set would be, for a given covering set, 
the sum of the lengths of the covering intervals.
We would then define the length of a set to be the infimum over all such covering bounds.
We call this ``length'' the {\em Lebesgue outer measure\/}. It is defined more precisely by:

\begin{eqnarray}
	\mu^*(S) & = & \inf_{\mathop{\cup}_{i=1}^\infty\limits I_i \subseteq S} \sum_{i=1}^\infty |I_i| 
\end{eqnarray}

This notion of measure (length) would not be of much use if the Lebesgue outer measure of
an interval was not its length.
It is not hard to show that it does indeed coincide with length on 
intervals -- and also finite collection of intervals.

Well what is the outer measure of the set $R$? We claim that the ``length'' 
(outer-measure) of the set of rationals on the interval $[0,1]$ is $0$.

Consider the coverings: $I_{(r_i - \frac{1}{\epsilon 2^i}, r_i + \frac{1}{\epsilon 2^i})}$. 
Then for any
$\epsilon > 0$, $R \subset \mathop{\cup}_{i=1}^\infty\limits I_{(r_i - \frac{1}{\epsilon 2^i}, r_i + \frac{1}{\epsilon 2^i})}$
Then $\forall \epsilon: \mu^*(R) \le 2 \epsilon$

Consequently, $\mu^*(R) = 0$. This approach will not work with the irrationals 
as they are not countable. In fact, any countably infinite cover of the 
irrationals with intervals must have a total length of at least $1$. 
If not, there would be some open interval  $[0,1]$, that is not covered. 
But any such open 
interval must contain an irrational; therefore all covers must have a total length 
of at least $1$. Since the cover 
$\{(0,1), (-\epsilon, \epsilon), (1 - \epsilon,1 + \epsilon)\}$
is a cover (for an infinite cover, let all other covers sets be set 
to $(0,1)$), with total length $1 + 2 \epsilon$.
Consequently, we see that $\mu^*(IR) = 1$.

With this we can define the Lebesgue Integral of the function: 
$f(x) = \chi_{R}(x)$ over the interval $[0,1]$ 
as $1 * \mu^*(R) = 1 * 0 = 0$.

In general, if a function can be written as a finite linear combination of 
characteristic functions: $h(x) = \sum_i^N c_i \chi_{E_i}(x)$,
then we would {\em define\/} the Lebesgue integral for such functions as:

\begin{eqnarray}
	\int_a^b h(x) \, dx & = & \sum_{i=1}^N\limits c_i \mu^*(E_i)
\end{eqnarray}
That is, we did what we said we would do; namely, we should compute the new integral 
of a ``simple'' function like the one above by taking its range values (the heights) 
and then multiplying by the ``width'' of the set of points that map to each 
of the heights. The widths of the sets are found by {\em squeezing\/} all 
of the ``air'' out.

Going further, we would try to define the Lebesgue Integral of an ``arbitrary''
function, $f$, as the limit of integrals of approximating ``simple'' 
functions  -- ones that are a finite linear combinations of characteristic 
functions with the property that they converge pointwise to $f$.

This is the initial plan. However, there is a problem with this approach. 
In the next section we provide an aside that describes a potential problem 
with the Lebesgue Outer Measure when used as a measure of ``length''.


\section{Outer Measure Aside: Measuring Area in the Real World}
An entrepreneur 
wanted to measure surfaces in the wild. He had heard that the 
ancient Greek mathematician Archimedes measured volumes of non-regular objects 
by placing them in a water basin. By observing the amount the water rises, he 
could measure the volume of water displaced and consequently, the volume of the object.

The entrepreneur wanted to do something similar to help measure the area of 
non-regular surfaces. He found that a 
certain oil when painted on a surface would accumulate at a fixed depth. By 
removing the oil (a proprietary process) and measuring the volume he could 
compute the area by dividing by this fixed depth. He called this method the 
``mu'' method and used the symbology $\mu(C)$ to be his measure of the surface 
area of an object $C$.

One day a friend of his pulled out his comb from his pocket and asked him to 
find the surface area. The technique seemed to work fine, it registered that 
the comb was a little more that half the area as the comb's container.

A few days later the friend came by with a new comb. It was actually a comb 
set, consisting of two combs that fit together perfectly to form a rectangle.
The teeth of the comb were much finer than the comb from a few days earlier. 
The retailer said is was much easier to store, you got a finer comb, and you 
get two combs for almost the price of one; and, better yet, they both 
fit in the same rectangular carrying case.

The entrepreneur applied his technique to the new combs. 
He found that, when fit together, they had the area of the rectangular carrying case.
He determined the area of this rectangle to be $A$. He then tried each of the 
combs separately only to find that each of their areas was also $A$!
It seems that due to the viscosity of the oil and the fine spacing between the 
teeth of the combs, the oil coating did not penetrate the fine crevasses of the 
comb, treating it as solid -- consequently, the measure of the area was 
that of the (enclosing) rectangle.


The entrepreneur had to admit that his technique failed on the combs; 
there were clearly limitations to his technique. 
But how would he describe the limitation? Basically, the entrepreneur wanted to 
be able to advertise that his measurement method could be used for all 
``reasonable'' objects. Exceptions would be objects that needed restoration 
that were broken into two pieces with extremely jagged edges.

What he wanted to say is that given an object, $C$ could be $\mu$ measured 
{\em if\/} $C$ was inside another object, $O$ and the following was true:

$\mu(C) + \mu(C^c) = \mu(O)$

Here, $C^c$ is the complementary set inside of $O$.

In some cases, it wasn't clear what the full object, $O$, consisted of. A better way 
to describe the limitation was to say that his measurement system was accurate 
when -- for {\em any\/} given bounding frame $X$ containing a set $C$ -- 
the measures of $C$ and 
its complement in $X$ added to the measure of $X$. That is,

$\mu(X) = \mu(X \cap C) + \mu(X \cap C^c)$

Unfortunately, this was not of much help -- in other words, you know the 
measurements work on reconstruction only after you've done them.
He stayed with his original caveat that the measurement works when some object has 
broken as long as the pieces are not ``too'' jagged.


\section{Back to Outer Measure}
It turns out that outer measure behaves much like the oil measurement system 
used in the last section. That is, there are sets, $C$, for which there is at 
least some, $X$ ("bounding box"), so that
$\mu^*(X) \neq \mu^*(X \cap C) + \mu^*(X \cap C^c)$

Sets like, $C$, are too jagged and its measure doesn't add properly.

We say that a set, $C$, is {\em measurable\/}
if for all sets $X$ the following is true:
\begin{eqnarray}
	\mu^*(X) & = & \mu^*(X \cap C) + \mu^*(X \cap C^c) \label{Carthy}
\end{eqnarray}
When a set is measurable we simply write $\mu(C)$ instead of $\mu^*(C)$.

What we're looking for is a requirement to be measurable -- meaning
outer measure works properly. By this we mean sets must be well behaved
in the sense that the length of a set and the length of its ``compliment'' 
must add up to their union. When you think about this statement it
seems clear that we very much need the ``lengths'' of sets and their compliments
to add properly before we go any further with an integration theory -- this 
is the least you expect/need from a new integration theory.

But why all the fuss about the intersection with ``any set $X$''.
Why not just say that $C$ and its compliment add up properly?
The problem is that the compliment of $C$ might have infinite length, because
the ambient space containing $C$ is infinite.
Saying that $C$ and its compliment add up correctly is the statement that 
the length of $C$ and the length of its compliment is equal to the length 
of the ambient space: length(C) + $\infty$ = $\infty$ -- a useless statement.
This won't distinguish any sets as being special; that is, it won't help
us filter out the badly behaving sets.

What we want to say, just like the combs from the
previous section, is that the measure of $C$ added with the measure of its 
``natural complement'' -- coming from a ``natural'' bounding box -- 
gives back the measure of the bounding box.
Without some restriction, the compliment could be 
an infinitely large set.

The problem now becomes: ``What is the 'natural' bounding box for the set $C$''?
We can get around this problem by requiring that it be true for {\em all\/} 
bounding boxes. And now, we see what ``X'' accomplishes in the definition of 
measurability, (\ref{Carthy}). The ``X'' in this definition is a stand-in for 
{\em all\/} bounding boxes; but this includes, in particular, the/any 
bounding boxes that might be considered ``natural''. 
Consequently, (\ref{Carthy}) provides an unambiguous definition to filter out 
sets that behave badly, while cleanly avoiding 
the problem of the definition of a set's bounding box.
However, we've solved one problem, but now have another. Yes, we have 
a clean definition, but can one use this in practice to show that a 
set is measurable? The definition is clean, but the burden to show a set is
measurable has very much increased!

We have now attempted to motivate the restriction of the length-of-sets 
measuring method called outer measure to so-call measurable sets via 
the filtering condition, (\ref{Carthy}).
This filter is called ``Carath\'eodory's Condition'',
and is introduced in most text books without much motivation. 
And yet it is the key idea that makes Lebesgue theory work. 
 
So, given that we restrict ourselves to these ``measurable'' sets it turns out that
the measure behaves in the way one might want. Namely.
\begin{eqnarray*}
	\mu(A) & = & \sum_{i=1}^N\limits \mu(A_i) \quad \text{when} \; 
		A = \mathop{\cup}_{i=1}^N\limits A_i \quad \text{with} \;  
		A_i \cap A_j = \emptyset \quad \text{if} \; i \neq j \\
	\mu(A) & = & \sum_{i=1}^\infty\limits \mu(A_i) \quad \text{when} \; 
		A = \mathop{\cup}_{i=1}^\infty\limits A_i \quad \text{with} \; 
		A_i \cap A_j = \emptyset \quad \text{if} \; i \neq j
\end{eqnarray*}

That is, the measure of a set that is composed of a collection of disjoint 
sets should be the sum of the measures of the disjoint sets. We need this to hold
for infinite sets as well because in any definition of integration we are going 
to need to take limits.
And, as you might imagine, the solution to determining when one can apply the 
outer measure effectively boils down to the above two behaviors.

It turns out that if we restrict ourselves to using only measurable sets, 
the resulting sets have the property that they are an 
{\em algebra\/} with respect to unions, intersections and complements. 
In fact, the set of measurable sets is a {\em $\sigma$-algebra\/}:
the measurable sets are closed under all finite or countably infinite unions, 
intersections and the complements of these.

Being an algebra, we can get a good handle on what sets are measurable if we 
can come up with a good {\em base\/} collection of sets that are {\em easily\/} shown
to be measurable. From this, we {\em automatically\/} know that a set is measurable 
-- no matter how complicated -- if it is built out of these base sets
through some (possibly infinite) combination of set operations.
From this, we can effectively bootstrap the set of measurable sets.
Once a complicated set is determined to be measurable, it is part of the 
algebra and can be used along with the base measurable sets -- and anything 
else determined to be in the algebra -- to construct other measurable sets. 
And we can continue this process ad infinitum.

Recapping, we have realized that our outer measure -- a ``natural'' way of 
measuring the length of sets -- is flawed. We found a way to restrict the 
usage to well behaved sets. {\em Unfortunately\/}, to make the restriction precise, 
we made the restriction verification very difficult to use. {\em Fortunately\/}, 
the resulting restricted sets are a {\em $\sigma$-algebra\/} and this helps 
save the day. We need only do the difficult work of verifying the restriction 
condition, (\ref{Carthy}), on simple base sets and then we can claim that 
sets formed by complicated set operations on the base set are automatically 
measurable sets. If we choose the simple base sets 
to be the collection of open and closed intervals, we have a collection that 
is easy to show satisfy, (\ref{Carthy}), and provide a wealth of complex sets
via set operations.

With a healthy collection of measurable sets we can define ``measurable functions``
as any function that is a simple function or is a point-wise limit of simple functions.
The connection to measurable sets is that the simple functions -- remember they 
are based on characteristic functions of sets -- use only use measurable sets in 
their construction.

The Lebesgue integral of these functions is the corresponding finite or infinite
sum of the integrals of their constituent simple functions. And again, the Lebesgue
integral of any base component of simple function -- a characteristic function --
is 1 * (outer measure (length) of the characteristic set).

Restricting ourselves to measurable sets does have a consequence: just like 
Riemann integration, we will not be able to integrate all functions, only 
measurable functions. But, such functions turn out to be a 
much larger and richer class of functions than the set of Riemann integrable functions.

Now, we are finally in a position to address the convergence theorems 
discussed at the beginning of this paper.


\section{Conclusion}

The path to creating a definition of a Lebesgue integral proceeds as mentioned 
in the previous sections -- use approximating functions that are finite linear
combinations of characteristic functions. Such combinations are called ``simple''
functions. They take the role that step functions played for Riemann integration. 

One way to create approximating simple functions is by chopping up the range 
of the function into evenly spaced (if the range is finite) intervals and then 
looking at the domain values of $f$ that map into this range interval. 
The set of values in the domain could, potentially, be a complicated set. 
But if they are all measurable, then one can create an approximating simple 
lower bound for the integral of $f$ from simple functions based on these sets. 
This turns out to work provided -- as you should imagine -- that the ``domain'' 
sets are measurable. This leads to restricting the integration to functions, $f$, 
that are ``measurable functions''. It turns out that all the usual functions you 
encountered in first year calculus as well as any piecewise continuous functions 
are measurable functions. But the set of measurable functions includes functions 
that are not Riemann integrable -- including the function we started out with, $f$, 
that was $1$ on the rationals and $0$ on the irrationals.

One can also show that the Lebesgue integral coincides with the Riemann integral 
for ``nice'' functions.

Finally, from this theory one does indeed get less stringent convergence 
criterion than the Riemann counterpart. One can have a criterion that avoids 
``uniform'' convergence. In fact, essentially, you get a convergence theorem 
that requires only that a sequence of functions converge point-wise to a function%
\footnote{In fact, one only requires that a sequence of functions converge point-wise 
to a function ``almost everywhere`` -- meaning everywhere except on a set of 
outer measure $0$.} and that the sequence of functions 
are ``dominated'' -- bounded above -- by a Lebesgue integrable function.

It should be noted that there is an analogy between Riemann and Lebesgue integration 
and the Rational and the Real number systems. In the same way that the Real number 
system ``completes'' the Rational number system. The Lebesgue integration 
theory ``completes'' the Riemann theory. In fact, with the Lebesgue theory 
one can create function spaces with a metric that makes them ``complete'' 
in the same sense as Real numbers; namely, Cauchy sequences always converge 
in that space. The function space $L_2$ consists of all functions $f$ 
(say over the interval $[0,1]$) that satisfy:

\begin{eqnarray}
\int_0^1 f^2(x) \, dx < \infty
\end{eqnarray}

Here, the integral is the Lebesgue integral. This space also has an inner 
product defined, for two functions $f, g \in L_2$ by

\begin{eqnarray}
\langle f, g \rangle \; \equiv \; \int_0^1 f(x)\, g(x)\, dx
\end{eqnarray}

And since it is a complete space, it is what is known as a {\em Hilbert Space\/}.
A Hilbert space is a vector space with an inner product that is complete with 
respect to the metric induced from its inner product. 
That is, Cauchy sequences converge to an element in this
space, just like they do for the finite dimensional Hilbert spaces: $R, R^2, R^3, \ldots$.
And, much of the behavior of the function space, $L_2$ is replicated from 
the finite dimensional $R^n$ spaces. This wouldn't be the case if we used
the Riemann integral for the inner product/metric. It would be like
working with the vector spaces $Q, Q^1, Q^2, \ldots$, where $Q$ is the set 
of rational numbers. The vector properties would be there, but not 
the important analytic properties -- and the notion of perpendicularity would
be missing.

We should mention that, technically, the function space $L_2$ does not 
consist of functions, it consists of 
equivalence classes of functions having the property
that they are identical except possibly on a set of measure $0$.

If this seems strange to you, it shouldn't. The rational numbers are, strictly 
speaking, equivalence classes of integer pairs.
For instance, the fraction that we write as '1/2' is really the 
equivalence class (set) of pairs of integers: 
\{(1,2), (2,4), (3, 6), (4,8), \ldots \}.
To be equivalent, any two pairs, $(a, b)$ and $(c, d)$ must satisfy: 
$a * d = b * c$. Also, note that each equivalence class of integer pairs --
representing a given rational number -- consists of an {\em infinite\/} number 
of integer pairs.

In the same way, the Real number $1$ is not the Rational number $1$, it is 
actually an equivalence class; namely, the set of all Cauchy sequences that 
converge to the Rational number $1$ -- the simplest of which is the 
sequence: $1, 1, 1, 1\ldots$ .


\end{document}

