\input macro

\parindent=0pt
\parskip=.5\baselineskip
\baselineskip = 1.1\baselineskip

\footline{\hss\tenrm\folio\hss}

\mantitle{A Balance Law Interpretation of the Fokker-Planck Equation}
{R. Scott McIntire}{Nov 22, 2002}

\subsection{Overview}

In this paper we use a balance law to derive the heat equation in one and 
several dimensions. We also use it to study the propagation of a probability
density function driven by a differential equation. Finally, we combine the 
two derivations and provide a derivation of the Fokker-Planck equation.


\section{Derivation of the Heat Equation}

We examine the problem of heat flow in $R^n$. First we look at the 
problem in one dimension and then see how it generalizes to arbitrary
dimension. The strategy is to write a balance law for the heat in 
an arbitrary region in $R^n$ based on the conservation of energy. 
This will lead to a local expression of heat flow;
namely, a partial differential equation. By relating heat to temperature, 
an equation for the temperature is obtained. 
The resulting partial differential equation for temperature is called 
{\it the heat equation\/}.


\subsection{The Heat Equation in One Dimension}
We derive a partial differential equation for the temperature distribution 
in a one dimensional material. To do this we 
derive a balance law for the heat contained in an arbitrary interval 
$[x_1,x_2]$ over an arbitrary time interval $[t_1, t_2]$. 
Let $h(t,x)$ be the 
heat per unit length at time $t$ and position $x$. Let 
$f(t,x)$ be the heat flux at time $t$ and position $x$. Where heat 
flux is defined as the rate at which heat flows
\footnote{\kern -2pt \raise 0.5ex \hbox{\dag}}{positive means 
increasing $x$.}. The increase (or decrease) in 
the heat of 
the interval $[x_1, x_2]$ over $[t_1, t_2]$ is equal to the integral of
the rate at 
which heat has moved into (or out of) the region through the boundaries.
Mathematically, this balance may be expressed as:
$$
\int_{x_1}^{x_2} h(t_2,x) - h(t_1,x) \; dx  = 
\int_{t_1}^{t_2} f(t, x_1) - f(t,x_2) \; dt
$$
Now, divide this expression by $1 / (t_2 - t_1)$ and take the 
limit as $t_2 \rightarrow t_1$. This gives:
$$
\eqalignno{
\int_{x_1}^{x_2} h_t(t_1,x) \; dx & = f(t_1,x_1) - f(t_1, x_2) & (1) \cr}
$$
This can be written as
$$
\eqalignno{
\fracd{d}{dt} \int_{x_1}^{x_2} h(t,x) \; dx \bigg|_{t=t_1} & 
= f(t_1,x_1) - f(t_1, x_2) 
& (2) \cr}
$$
Equation (2) can be interpreted as an instantaneous balance law for the 
heat contained in the interval $[x_1, x_2]$. It states that the rate of 
change of the heat in the interval $[x_1, x_2]$ is equal to the rate 
at which heat enters through the boundary.


By the {\it the Fundamental Theorem of Calculus\/}, 
$\int_{x_1}^{x_2} f_x(t_1, x) \, dx = f(t_1, x_2) - f(t_1, x_1)$.
Therefore, bringing the right hand side to the left hand side in equation 
(1) we may write{$\vphantom |$}:
$$
\int_{x_1}^{x_2} h_t(t_1,x) + f_x(t_1,x) \; dx = 0
$$
If we {\it assume\/} that the integrand is continuous, then since the 
interval $[x_1, x_2]$ is arbitrary, the integrand must 
vanish for all $x$; that is,
$$
h_t(t_1,x) + f_x(t_1,x) = 0
$$
Finally, since $t_1$ is arbitrary the above must be true for all times $t$;
so that,
$$
h_t(t,x) + f_x(t,x) = 0
$$
To get an equation for the temperature, $u$, we must relate it to 
the heat and the heat flux. To do so, we must make 
{\it assumptions\/} about properties of the material.
The simplest assumption about the heat is that it is 
proportional to temperature:
$h(t,x) = c \, u(t,x)$. 
This proportionality constant, $c > 0$, is called the specific heat. 
The simplest assumption about the heat flux is that it is proportional 
to the temperature gradient:
$f(t,x) = - k u_x(t,x)$ with $k > 0$. 
Therefore, we may write:
$$
c \, u_t - k u_{xx} = 0
$$
Letting $\kappa = k / c$ we obtain the one dimensional heat equation 
with diffusion constant $\kappa$:
$$
\fracd{\partial u}{\partial t} = \kappa \fracd{\partial^2 u}{\partial^2 x}
$$
The larger the value of $\kappa$, the more rapid the diffusion. Since 
$\kappa$ is made from two constants, we give an interpretation to understand
their role. Consider a bar made of different materials and different 
thicknesses. Let us assume that the differing materials 
all have the same heat capacity, but different heat flux rate constants.
Then we have two independent 
controls: the thickness of the bar controls the value of the 
specific heat, $c$; and, the material controls the heat flux rate, $k$. 

Under these conditions, we can get a better physical 
understanding of the influence 
of $c$ and $k$ on $\kappa$. If the bar is thicker at a certain spot, 
it can absorb more heat and 
therefore, is less diffusive there. 
If the bar is made of a material that is more 
insular at a certain spot, the heat flux rate constant goes down and 
therefore, the material is less diffusive there.

We note that $c$ and $k$ may depend on $x$.
In this case we obtain a generalized heat equation:
$$
\fracd{\partial u}{\partial t} = \fracd{1}{c(x)} 
\fracd{\partial}{\partial x} \bigg(k(x) \fracd{\partial u}{\partial x}\bigg)
$$
Technically, to have similar properties of the heat equation, $c(x)$
and $k(x)$ should satisfy(for all $x$):

\beginEnum
\enum{$c(x) > c_0$ for some $c_0 > 0$.}
\enum{$k(x) > k_0$ for some $k_0 > 0$.}
\enum{$c(x)$ is continuous and $k(x)$ is continuously 
differentiable with respect to $x$.}
\endEnum

\subsection{The Heat Equation in Higher Dimensions}
We derive a partial differential equation for the temperature distribution 
of a material in $R^n$. To do this we will mimic each of the steps 
in the last section. We start by 
deriving a balance law for the heat contained in an arbitrary region
$\Omega$ over an arbitrary time interval $[t_1, t_2]$. 
Let $h(t,{\bf x})$ be the 
heat per unit volume at time $t$ and position ${\bf x}$. Let 
${\bf f}(t,{\bf x})$ be the heat flux at time $t$ and position ${\bf x}$. 
Where heat flux is defined as a vector describing the rate at which heat flows
per area\footnote{\kern 0.5pt \raise 0.5ex \hbox{\dag}}
{By area we mean  the n-1 dimensional area.}.
The increase (or decrease) in 
the heat of 
the region $\Omega$ over $[t_1, t_2]$ is equal to the integral of the 
rate at which heat has moved into (or out of) the region through the 
boundaries.
Mathematically, this may be expressed as:
$$
\int_{\Omega} h(t_2, {\bf x}) - h(t_1, {\bf x}) \; d{\bf x} = -
\int_{t_1}^{t_2} \!\! 
\mathop{\underbrace{\oint_{\partial \Omega} 
\mathop{\overbrace{{\bf f}(t, {\bf x}) \cdot {\bf n}}}\limits^
{\hbox{\small Heat flow rate per unit area leaving  \hbox{$\Omega$}}}
\; d{\bf S}}}\limits_{\hbox{\small Heat flow rate leaving 
\hbox{$\Omega$}}} \; dt
$$
As in the one dimensional case, divide by $1 / (t_2 - t_1)$ 
and take the limit as 
$t_2 \rightarrow t_1$. This gives:
$$
\eqalignno{
\int_{\Omega} h_t(t_1, {\bf x}) \; d{\bf x} & = -
\oint_{\partial \Omega} {\bf f}(t_1, {\bf x}) \cdot {\bf n} 
\; d{\bf S} & (3) \cr }
$$
This can be written as
$$
\eqalignno{
\fracd{d}{dt} \int_{\Omega} h(t, {\bf x}) \; d{\bf x} \bigg|_{t=t_1} & = -
\oint_{\partial \Omega} {\bf f}(t_1, {\bf x}) \cdot {\bf n} 
\; d{\bf S} & (4) \cr}
$$
Equation (4) can be interpreted as an instantaneous balance law for the 
heat contained in the region $\Omega$. It states that the rate of 
change of the heat in the region $\Omega$ is equal to the rate 
at which heat enters through the boundary.


By the {\it the Fundamental Theorem of Calculus\/}, 
$\int_{\Omega} \nabla \cdot {\bf f}(t_1, {\bf x}) \, d{\bf x} = 
\oint_{\partial \Omega} {\bf f}(t_1, {\bf x}) \cdot {\bf n} \, d{\bf S}$.
Therefore, bringing the right hand side to the left hand side in equation 
(3) we may write{$\vphantom |$}:
$$
\int_{\Omega} h_t(t_1, {\bf x}) + \nabla \cdot {\bf f}(t_1, {\bf x}) 
\; d{\bf x} = 0
$$
If we {\it assume\/} that the integrand is continuous, then since the 
region $\Omega$ is arbitrary, the integrand must vanish for all $x$;
that is,
$$
h_t(t_1, {\bf x}) + \nabla \cdot {\bf f}(t_1, {\bf x}) = 0
$$
Finally, since $t_1$ is arbitrary the above must be true for all times $t$;
so that,
$$
h_t(t, {\bf x}) + \nabla \cdot {\bf f}(t, {\bf x}) = 0
$$
To get an equation for the temperature, $u$, we must relate it to 
the heat and the heat flux. To do so, we must make 
{\it assumptions\/} about the properties of the material. The simplest 
assumption about the heat is that it is proportional to
temperature:
$h(t,{\bf x}) = c \, u(t,{\bf x})$. 
As in the one dimensional case, the proportionality constant, $c > 0$, 
is called the specific heat. 
The simplest assumption about the heat flux is that it is proportional 
to the temperature gradient:\footnote{\kern 1pt \raise 0.5ex \hbox{\ddag}}
{That is, the {\it direction\/} of heat 
flow should be in the opposite direction of maximum change of temperature;
and, the {\it magnitude\/} of the heat flow should be proportional to the 
magnitude of the maximum change in the temperature.} 
${\bf f}(t,{\bf x}) = - K \nabla u(t,{\bf x})$ with $K > 0$. 
Therefore, we may write:
$$
c \, u_t(t, {\bf x}) - K \nabla \cdot \nabla u(t, {\bf x}) = 0
$$
Letting $\kappa = K / c$ and noting that $\Delta = \nabla \cdot \nabla$, 
we obtain the n-dimensional heat equation with diffusion constant $\kappa$.
$$
\fracd{\partial u}{\partial t} =  \kappa \, \Delta u 
$$
We note that $c$ and $K$ may depend on ${\bf x}$ and we note $K$ could be 
a matrix. In this case we obtain a generalized heat equation:
$$
\fracd{\partial u}{\partial t} = \fracd{1}{c({\bf x})} 
\nabla \cdot \bigg(K({\bf x}) \nabla u \bigg)
$$
Technically, to have similar properties of the heat equation, $c({\bf x})$
and $K({\bf x})$ should satisfy(for all ${\bf x}$):

\beginEnum
\enum{$c({\bf x}) > c_0$ for some $c_0 > 0$;}
\enum{$K({\bf x})$ should be a positive definite symmetric matrix;}
\enum{The smallest eigenvalue of $K({\bf x})$ is greater than $k_0$ for some
$k_0 > 0$;}
\enum{$c({\bf x})$ is continuous and $K({\bf x})$ is 
continuously differentiable with respect to ${\bf x}$.}
\endEnum


When $K({\bf x})$ is a matrix, the heat flux vector $f(t,{\bf x})$ will not, 
in general, point in the direction opposite of the maximum 
temperature increase. This may be interpreted physically as having a material
that has a different heat flow rate in different directions. In this case,
the heat flux vector, representing the heat flow path, will be modified 
by these material differences and will not necessarily flow along 
the temperature gradient.


\section{Propagation of Probability Density Functions}
In this section, we wish to derive equations for the propagation of 
probability density functions driven by systems of differential equations.

\subsection{A first Order System}
Given a first order system of differential equations of the form:
$$
\eqalignno{
\dot {\bf x} =  {\bf f}(t, {\bf x}) & & (5a) \cr
{\bf x}(0) = {\bf x_0}  & & (5b) \cr }
$$
let ${\cal X}(t; {\bf x_0})$ be the solution; 
which is assumed to be invertible 
on the interval $[0,T)$.
That is, for each $t \in [0,T)$, ${\cal X}(t; \cdot): R^n \mapsto R^n$ 
is invertible. If $X$ is a random variable with 
probability density function $P_X({\bf x})$, then we wish to compute 
the probability density function of the random variable induced by 
applying the transformation 
${\cal X}(t; {\bf \cdot})$ to the random variable $X$.
This will be done by writing down a balance law for a 
suitable probabilistic expression. As in previous sections, we will not 
obtain a formula for the induced probability density function; rather, 
we will produce a partial differential equation describing its evolution.

\subsection{A Balance Law Computation}
Given an arbitrary region, $\widehat \Omega$, and an arbitrary time 
$t \in [0,T)$, 
the probability contained in $\widehat \Omega$ 
will remain the same as the region is transformed either forward or 
backward in time 
under the differential equation (5a-b). Therefore, there is a one parameter 
family of regions $\Omega_\tau$ such that $\Omega_t$ = $\widehat \Omega$ 
and $\Omega_0 = \Omega$. That is, $\Omega$ morphs into $\widehat \Omega$ 
under the action of the differential equation (5a-b).
Let 
$p(t,{\bf x}) = P_{{\cal X}(t;X)}({\bf x})$ be the probability
density function of the induced random variable ${\cal X}(t; X)$. 
The conservation of probability of $P_{{\cal X}(t, X)}(\Omega_t)$ 
is expressed mathematically as an instantaneous balance law:
$$
\eqalignno{
\fracd{d}{d\tau} \int_{\Omega_\tau} p(\tau,{\bf x}) \; d{\bf x} & 
= 0 & (6) \cr}
$$
To compute the derivative we change the integration variable so as to 
keep the region constant. To this end, let 
${\bf x} = {\cal X}(\tau, {\bf z})$. The integral becomes:
$$
\fracd{d}{d\tau} \int_{\Omega} 
p(\tau,{\cal X}(t, {\bf z})) \big| 
\frac{\partial {\cal X}(\tau,{\bf z})}
{\partial {\bf z}} \big| \; d{\bf z} = 0
$$
which is 
$$
\eqalignno{
\int_{\Omega} \big( p_{\tau}(\tau, {\cal X}(\tau, {\bf z})) + 
\nabla p(\tau, {\cal X}(\tau, {\bf z})) \cdot \dot 
{\cal X}(\tau, {\bf z})
  \big) 
\big| \frac{\partial {\cal X}(\tau,{\bf z})}
{\partial {\bf z}} \big| + 
p(\tau, {\cal X}(\tau, {\bf z})) \frac{d}{d\tau}  \big| 
\frac{\partial {\cal X}(\tau,{\bf z})}
{\partial {\bf z}} \big| \; d{\bf z} & = 0 & (7) \cr }
$$
It turns out that the derivative, $\frac{d}{d\tau} 
\big| \frac{\partial {\cal X}(\tau,{\bf z})}
{\partial {\bf z}} \big|$, may be written 
as\footnote{\kern 1pt \raise 0.5ex \hbox{\dag}}{This is a consequence of 
Louisville's Theorem.}
$$
\frac{d}{d\tau} 
\big| \frac{\partial {\cal X}(\tau,{\bf z})}
{\partial {\bf z}} \big| =
\big| \frac{\partial {\cal X}(\tau,{\bf z})}
{\partial {\bf z}} \big| \; {\rm tr}\big(\nabla {\bf f}
(\tau, {\cal X}(\tau, {\bf z}))\big) =
\big| \frac{\partial {\cal X}(\tau,{\bf z})}
{\partial {\bf z}} \big| \; 
\nabla \cdot {\bf f}(\tau, {\cal X}(\tau, {\bf z}))
$$
Therefore, (7) becomes
$$
\eqalignno{
\int_{\Omega} \bigg( p_\tau(\tau, {\cal X}(t, {\bf z})) + 
\nabla p(\tau, {\cal X}(\tau, {\bf z})) \cdot 
{\bf f}(\tau, {\cal X}(\tau, {\bf z}))
+ p(\tau, {\cal X}(\tau, {\bf z}))
\nabla \cdot {\bf f}(\tau, {\cal X}(\tau, {\bf z}))
\bigg) 
\big| \frac{\partial {\cal X}(\tau,{\bf z})}
{\partial {\bf z}} \big| \; d{\bf z} & = 0 & (8) \cr }
$$
Changing coordinates back, (8) becomes
$$
\eqalign{
\int_{\Omega_\tau} p_{\tau}(\tau, {\bf x}) + \nabla p(\tau, {\bf x}) 
\cdot {\bf f}(\tau, {\bf x}) 
+ p(\tau, {\bf x}) \nabla \cdot {\bf f}(\tau, {\bf x}) \; d{\bf x} & = 0 \cr }
$$
This may be written as
$$
\eqalignno{
\int_{\Omega_\tau} p_\tau(\tau, {\bf x}) + \nabla \cdot 
\big( p(\tau, {\bf x}) {\bf f}(\tau, {\bf x}) \big)
 \; d{\bf x} & = 0 & (9) \cr }
$$
Equation (9) is true for all $\tau \in [0, T)$, 
in particular it is true for $\tau = t$.
If we assume that the integrand in (9) is continuous and since 
$\widehat \Omega = \Omega_t$ and $t$ are both arbitrary, we have for all 
$t \in [0,T)$
and for all ${\bf x}$
$$
\eqalignno{
p_t(t, {\bf x}) + \nabla \cdot 
\big( p(t, {\bf x}) {\bf f}(t, {\bf x})\big) & = 0 & (10a) \cr
p(0, {\bf x})  = P_X({\bf x})& & (10b) \cr} 
$$
In coordinates this is written as
$$
\eqalign{
\fracd{\partial p(t, {\bf x})}{\partial t} + 
\sum_{i=1}^n \fracd{\partial}{\partial x_i} 
\big( p(t, {\bf x}) {\bf f}(t, {\bf x})\big) & = 0  \cr
p(0, {\bf x})  = P_X({\bf x})&  \cr} 
$$
Equations (10a-b) constitute a first order partial differential 
equation describing the evolution of the probability density function 
$P_X({\bf x})$ with respect to the differential equation (5a-b).

\subsection{A Second Order System}
Many physical systems are more naturally described by a system of second 
order differential equations than a first order system. In this section, 
we derive an equation governing the evolution of a probability 
density function driven by a second order system.
Consider the second order system of differential equations of the form:
$$
\eqalignno{
\ddot {\bf x} =  {\bf f}(t, {\bf x}, \dot {\bf x}) & & (11a) \cr
{\bf x}(0) = {\bf x_0} & & (11b)  \cr
\dot {\bf x}(0) = {\bf v_0} & & (11c)  \cr} 
$$
This may be written as a first order system as follows: 
Let ${\bf v} = \dot {\bf x}$, then (11a-c) may be written as
$$
\eqalignno{
\dot {\bf x}  & = {\bf v} &  (12a) \cr
\dot {\bf v}  & =  {\bf f}(t, {\bf x}, {\bf v}) &  (12b) \cr
{\bf x}(0)  & = {\bf x_0}  & (12c) \cr 
{\bf v}(0)  & = {\bf v_0}  & (12d) \cr }
$$
We can use the result of the last section to write a formula for the 
propagation of the probability density function. Let $P_{X,V}({\bf x},{\bf v})$
be the joint probability density function for random variables $X$ and $V$.
Let $p(t,{\bf x},{\bf v}) = 
P_{{\cal X}(t; X, Y), {\cal V}(t; X, Y)}({\bf x}, {\bf v})$ 
denote the probability density function of the pair, $(X, Y)$, transformed by
the solution of (12a-d) at time $t$.
From the previous section, $p$ satisfies 
\footnote{\kern -3pt \raise 0.5ex \hbox{\dag}}
{This follows from the identity  
$\nabla \cdot (\varphi{\bf w}) = \nabla \varphi \cdot {\bf w}
+ \varphi \, {\rm tr}\, \nabla {\bf w}$.}
$$
p_t(t, {\bf x}, {\bf v}) + \pmatrix{ \nabla_{\bf x}p(t, {\bf x}, {\bf v}) 
\enspace , & \!\!\!\! \nabla_{\bf v}p(t, {\bf x}, {\bf v})}
\cdot  \pmatrix{{\bf v} \cr {\bf f}(t, {\bf x}, {\bf v})} 
+ p(t, {\bf x}, {\bf v}) 
\; {\rm tr}\pmatrix{0 & I \cr \nabla_{\bf x}{\bf f}(t, {\bf x}, {\bf v}) 
& \nabla_{\bf v} {\bf f}(t, {\bf x}, {\bf v}) \cr} = 0
$$
Simplifying and adding the initial condition this becomes
$$
\eqalignno{
p_t(t, {\bf x}, {\bf v}) + \nabla_{\bf x}p(t, {\bf x}, {\bf v}) 
\cdot {\bf v} + 
\nabla_{\bf v}p(t, {\bf x}, {\bf v}) \cdot 
{\bf f}(t, {\bf x}, {\bf v}) + p(t, {\bf x}, {\bf v}) \nabla_{\bf v} \cdot
{\bf f}(t, {\bf x}, {\bf v}) & = 0 & (13a) \cr
p(0, {\bf x}, {\bf v}) = P_{X,V}({\bf x}, {\bf v}) & & (13b) \cr }
$$
Or,
$$
\eqalignno{
p_t(t, {\bf x}, {\bf v}) + \nabla_{\bf x}p(t, {\bf x}, {\bf v}) 
\cdot {\bf v} + 
\nabla_{\bf v} \cdot \big( p(t, {\bf x}, {\bf v}) 
{\bf f}(t, {\bf x}, {\bf v}) \big) 
 & = 0 & (14a) \cr
p(0, {\bf x}, {\bf v}) = P_{X,V}({\bf x}, {\bf v}) & & (14b) \cr }
$$
In coordinates this is written as
$$
\eqalignno{
\fracd{\partial p(t, {\bf x}, {\bf v})}{\partial t} + 
\sum_{i=1}^n \fracd{\partial p(t, {\bf x}, {\bf v})}{\partial x_i} v_i +
\sum_{i=1}^n \fracd{\partial}{\partial v_i} 
\big(p(t, {\bf x}, {\bf v}) f_i(t, {\bf x}, {\bf v}) \big) 
& = 0  & (15a) \cr
p(0, {\bf x}, {\bf v}) = P_{X,V}({\bf x}, {\bf v}) &  & (15b) \cr }
$$

\subsection{The Fokker-Planck Equation I}
We now consider the evolution of a probability density function 
driven primarily by the differential system (5a-b), but which incorporates
the dynamics of diffusion as well. 
Such a governing equation is called a Fokker-Planck 
equation. If $X$ is a random variable and ${\cal X}(t, {\bf x})$ 
is the solution of (5a-b), then let 
$p(t, {\bf x}) = P_{{\cal X}(t, X)}({\bf x})$
be the probability density function for the random variable $X$ transformed 
by ${\cal X}(t, \cdot)$. As before, 
let $\widehat \Omega$ be an arbitrary region in $R^n$ and let 
$\Omega_\tau$ parameterize a family 
of regions such that $\Omega_0 = \Omega$ and $\Omega_t = \widehat \Omega$.

Instead of equation (6) we assume that 
$$
\eqalignno{
\fracd{d}{d\tau} \int_{\Omega_\tau} p(\tau, {\bf x}) \; d{\bf x} & = 
\oint_{\partial \Omega_\tau} \big(A({\bf x}) \nabla p(\tau, {\bf x}) \big)
\cdot {\bf n} \; d{\bf S} & (16) \cr}
$$
In the simplest case, $A({\bf x}) = I$, equation (16) says that the 
changing regions, $\Omega_\tau$, allow diffusion based on the 
probability gradient on the boundary. This is the same mechanism as 
the heat equation. In the general case, the diffusion behaves as 
the diffusion in the generalized heat equation.

We have already done the computation for the left hand side of (16), so we 
may write
$$
\eqalignno{
\int_{\Omega_\tau} p_\tau(\tau, {\bf x}) + \nabla \cdot 
\big( p(\tau, {\bf x}) {\bf f}(\tau, {\bf x}) \big) \, d{\bf x} & = 
\oint_{\partial \Omega_\tau} \big(A({\bf x}) \nabla p(\tau, {\bf x}) \big)
\cdot {\bf n} \; d{\bf S} & (17) \cr}
$$
Using the {\it Fundamental Theorem of Calculus\/} on the right hand side 
of (17) and combining with the left hand side, we have
$$
\eqalignno{
\int_{\Omega_\tau} p_\tau(\tau, {\bf x}) + \nabla \cdot 
\big( p(\tau, {\bf x}) {\bf f}(\tau, {\bf x}) \big) -
\nabla \cdot \big( A({\bf x}) \nabla p(\tau, {\bf x}) \big) \, d{\bf x}
& = 0 & (18) \cr}
$$
Since this is true for all $\tau \in [0, t]$; and, $t$ and 
$\Omega_t = \widehat \Omega$ are arbitrary, we may write
the Fokker-Planck equation governing the evolution of 
$P_X({\bf x})$\footnote{\raise 0.5ex \hbox{\dag}}
{We make the same assumptions about $A({\bf x})$ 
(in this section and the next) that were made when 
deriving the heat equation in n dimensions:
\beginEnum
\enum{$A({\bf x})$ should be a positive definite symmetric matrix;}
\enum{The smallest eigenvalue of $A({\bf x})$ is greater than $a_0$ for some
$a_0 > 0$;}
\enum{$A({\bf x})$ is continuously differentiable with respect to ${\bf x}$.}
\endEnum}.
$$
\eqalignno{
p_t(t, {\bf x}) + \nabla \cdot 
\big( p(t, {\bf x}) {\bf f}(t, {\bf x}) \big)  & =
\nabla \cdot \big( A({\bf x}) \nabla p(t, {\bf x}) \big)  & (19a) \cr
p(0, {\bf x}) = P_X({\bf x}) & & (19b) \cr}
$$
In coordinates this is written as
$$
\eqalignno{
\fracd{\partial p(t, {\bf x})}{\partial t} +
\sum_{i=1}^n \fracd{\partial}{\partial x_i} \bigg(p(t, {\bf x}) 
f_i(t, {\bf x}) \bigg) & = 
\sum_{i=1}^n \fracd{\partial}{\partial x_i} 
\bigg(
\sum_{j=1}^n A({\bf x})_{ij} \fracd{\partial p(t, {\bf x})}{\partial x_j} 
\bigg)
& (20a) \cr
p(0, {\bf x}) = P_X({\bf x}) & & (20b) \cr }
$$

\subsection{The Fokker-Planck Equation II}
We now consider the evolution of a probability density function 
driven primarily by the second order differential system (12a-d), 
but which incorporates the dynamics of diffusion as well. 
If $X$ and $V$ are random variables and 
$\{{\cal X}(t, {\bf x}, {\bf v})~,~{\cal V}(t, {\bf x}, {\bf v})\}$ 
is the pair of functions representing the solution of (11a-b), then 
let $p(t, {\bf x}, {\bf v}) = 
P_{{\cal X}(t, X), {\cal V}(t, V)}({\bf x}, {\bf v})$
be the joint probability density function for the 
random variables $\{X, V\}$ transformed 
by $\{{\cal X}(t, \cdot)~,~{\cal V}(t, \cdot)\}$. 
As before, let $\widehat \Omega$ be an arbitrary region in $R^n {\rm x} R^n$, 
and let 
$\Omega_\tau$ parameterize a family 
of regions in $R^n {\rm x} R^n$ such that 
$\Omega_0 = \Omega$ and $\Omega_t = \widehat \Omega$.

We assume, as in the previous section that the primary dynamics of the 
evolution of $p(t, {\bf x}, {\bf v})$ is the differential system (12a-d).
Additionally, we assume that probability diffuses through the ${\bf v}$ 
component of the regions 
$\Omega_\tau$ based {\it only\/} on the gradient of $p$ 
with respect to ${\bf v}$.
That is, the instantaneous balance law (6) becomes 
$$
\eqalignno{
\fracd{d}{d\tau} \int_{\Omega_\tau} p(\tau,{\bf x}, {\bf v}) \; 
d({\bf x} \hbox{\small x} {\bf v}) & 
= \oint_{\partial \Omega_\tau} \pmatrix{ 0 \! \! &, \;
A({\bf x}, {\bf v}) \nabla_{\bf v} 
p(\tau, {\bf x}, {\bf v})} \cdot \pmatrix{{\bf n_x} \cr
{\bf n_v}} \; d({\bf S_x} \raise 1pt \hbox{\small x} {\bf S_v}) & (21) \cr}
$$
The left hand side has been previously calculated; so, using the 
{\it Fundamental Theorem of Calculus\/} on the right hand side, we may 
write
$$
\eqalignno{
\int_{\Omega_\tau} p_\tau(\tau,{\bf x}, {\bf v}) +
\nabla_{\bf x} p(\tau, {\bf x}, {\bf v}) \cdot {\bf v} + 
\nabla_{\bf v} \cdot \big( p(\tau, {\bf x}, {\bf v}) 
{\bf f}(\tau, {\bf x}, {\bf v}) \big) & & \cr
 - \nabla_{\bf v} \cdot \big(A({\bf x}, {\bf v}) \nabla_{\bf v} 
p(\tau, {\bf x}, {\bf v})\big) \; d({\bf x} \hbox{\small x} {\bf v})
 & = 0 & (22) \cr}
$$
Again, $\widehat \Omega = \Omega_t$ and $t$ arbitrary imply that the 
Fokker-Planck equation governing the evolution of 
$P_{X,V}({\bf x}, {\bf v})$ is
$$
\eqalignno{
p_t(t,{\bf x}, {\bf v}) +
\nabla_{\bf x} p(t, {\bf x}, {\bf v}) \cdot {\bf v} + 
\nabla_{\bf v} \cdot \big( p(t, {\bf x}, {\bf v}) 
{\bf f}(t, {\bf x}, {\bf v}) \big) 
  & = \nabla_{\bf v} \cdot \big(A({\bf x}, {\bf v}) \nabla_{\bf v} 
p(t, {\bf x}, {\bf v})\big)  & (23a) \cr
p(0, {\bf x}, {\bf v}) = P_{X,V}({\bf x}, {\bf v}) & & (23b) \cr}
$$
Recalling that the coordinates for the right hand side have been written 
in the last section we may write the coordinate version of this 
Fokker-Planck equation as 
$$
\eqalignno{
\fracd{\partial p(t, {\bf x}, {\bf v})}{\partial t} +
\sum_{i=1}^n \fracd{\partial p(t, {\bf x}, {\bf v})}{\partial x_i}
v_i  + \sum_{i=1}^n 
\fracd{\partial}{\partial v_i} 
\big(p(t, {\bf x}, {\bf v}) f_i(t, {\bf x}, {\bf v}) \big)
& = 
\sum_{i=1}^n \fracd{\partial}{\partial v_i} 
\bigg(\sum_{j=1}^n A({\bf x}, {\bf v})_{ij} 
\fracd{\partial p(t, {\bf x}, {\bf v})}{\partial v_j} \bigg)
  & (24a) \cr
p(0, {\bf x}, {\bf v}) = P_{X,V}({\bf x}, {\bf v}) & & (24b) \cr }
$$
In the case when $A({\bf x}, {\bf v}) = \kappa {\cal A}$, 
${\cal A}$ a constant positive definite symmetric matrix, (24a-b) becomes
$$
\eqalignno{
\fracd{\partial p(t, {\bf x}, {\bf v})}{\partial t} +
\sum_{i=1}^n \fracd{\partial p(t, {\bf x}, {\bf v})}{\partial x_i}
v_i  + \sum_{i=1}^n 
\fracd{\partial}{\partial v_i} 
\big(p(t, {\bf x}, {\bf v}) f_i(t, {\bf x}, {\bf v}) \big)
 & = \kappa 
\sum_{i=1}^n 
\sum_{j=1}^n {\cal A}_{ij} 
\fracd{\partial^2 p(t, {\bf x}, {\bf v})}{\partial v_i \; \partial v_j}
  & (25a) \cr
p(0, {\bf x}, {\bf v}) = P_{X,V}({\bf x}, {\bf v}) & & (25b) \cr }
$$
If $A({\bf x}, {\bf v})$ is specialized further so that ${\cal A}$ 
is the identity matrix, then (25a-b) becomes
$$
\eqalignno{
\fracd{\partial p(t, {\bf x}, {\bf v})}{\partial t} +
\sum_{i=1}^n \fracd{\partial p(t, {\bf x}, {\bf v})}{\partial x_i}
v_i  + \sum_{i=1}^n 
\fracd{\partial}{\partial v_i} 
\big(p(t, {\bf x}, {\bf v}) f_i(t, {\bf x}, {\bf v}) \big)
& = \kappa 
\sum_{i=1}^n 
\fracd{\partial^2 p(t, {\bf x}, {\bf v})}{\partial^2 v_i}
  & (26a) \cr
p(0, {\bf x}, {\bf v}) = P_{X,V}({\bf x}, {\bf v}) & & (26b)  \cr }
$$
In the one dimensional case, this is further simplified to
$$
\eqalignno{
\fracd{\partial p(t, x, v)}{\partial t} + 
\fracd{\partial p(t , x, v)}{\partial x} v +
\fracd{\partial}{\partial v} 
\big( p(t, x, v) f(t, x, v) \big) & = 
\kappa \fracd{\partial^2 p(t, x, v)} {\partial^2 v} & (27a) \cr
p(0, x, v) = P_{X,V}(x, v) & & (27b) \cr }
$$
\bye


