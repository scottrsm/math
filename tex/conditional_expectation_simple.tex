%% Needed for the postscript graphics.
%%\input epsf

%% My macros.
\input macro

%% The length of a page is 10in.
%%\vsize=10in

%\font\helv = phvr
%\font\times = ptmr
%\font\zap = pzcmi at 12pt
%\font\pal = pplro at 12pt

%% Don't indent paragraphs.
\parindent=0pt
\parskip=.5\baselineskip
\baselineskip = 1.1\baselineskip

\footline{\hss\tenrm\folio\hss}

\mansubtitle{Measure Theoretic Conditional Expectation}
        {in an Elementary Setting}
{R. Scott McIntire}{Sep 6, 2024}

\section{Overview}
The measure theoretic approach to conditional expectation can be confusing
when compared to the traditional approach -- especially in a discrete 
setting. In what follows we go through a conditional 
expectation problem within a discrete and familiar setting in an attempt 
to reduce this confusion. In the example shown, we show explicitly that the 
conditional expectation function is non-measurable in the origin 
measure space.


\section{Elementary Probability Example Using Measure Theory}
Let $X = \{D_1, D_2, D_3, D_4, D_5, D_6\}$ and define a function $P$ by 
$P(D_i) = \frac{1}{6}$, for $i\in \{1,2,3,4,5,6\}$. The intent is that $P$ will 
become a probability measure for the space we construct.
Let ${\cal E} = 2^X$ be the sigma  algebra 
consisting of the power set of $X$. We extend $P$ for every element in the sigma algebra.
Since the sigma algebra consists of all sets 
we need an assignment for an arbitrary set, $A$. 
The assignment is $P(A) = \frac{|A|}{6}$; that is the cardinality of the set divided by 6.
We now have a measure space; in fact, a probability space: $(P,X,{\cal E})$.
Note that for a probability space we need an event space, $X$, a sigma algebra of sets 
(in the discrete case just an algebra), and a function $P$ which takes 
elements of the sigma algebra to [0,1] with the property that 
$$
P\bigl(\mathop{\bigcup}_{i=1}^N A_i\bigr) = \sum_{i=1}^N P(A_i) 
\quad {\rm when}\;  A_k \cap A_j = \emptyset \quad k\neq j
$$

We now consider a random variable from which we will get a sub-sigma algebra.
Let 
$$
g(D_i) = \cases{0 & if $i$ is even;\cr
1 & if $i$ is odd \cr 
}
$$
In a discrete space the sigma algebra generated from $g$ is the algebra of sets generated from 
the sets:
$g^{-1}(0), g^{-1}(1), g^{-1}(a)$, for $a \neq 0,1$. It is not too hard to see that 
${\cal F} = \{\emptyset, \{1,3\}, \{2,4,6\}, \{1,2,3,4,5,6\}\}$.

For a discrete space a function, $f$, is measurable with respect to a sigma algebra if 
$f^{-1}(a)$ is an element in the sigma algebra for all $a\in (-\infty, \infty)$.
This has implications for the sigma algebra ${\cal F}$. 

The claim is that any function 
which is measurable over ${\cal F}$ has the property that $f$ is constant on the sets 
$\{1,3\}$ and $\{2,4,6\}$. More generally, we claim that any measurable function $f$ 
in ${\cal F}$ must be constant on the {\it minimal\/} elements of the algebra 
--  elements which have no non-trivial subsets.%
\footnote{\kern 0.5pt \raise 0.5ex \hbox{\dag}}{By non-trivial subset we mean a subset that is not the empty set.}

To see this suppose that $f(1)$ differs from $f(3)$.
Then $\{1, 3\} \cap f^{-1}(f(1))$ is a non trivial subset of $\{1,3\}$ as is $\{1, 3\} \cap f^{-1}(f(3))$.
These two sets must differ since $f(1)$ and $f(3)$ differ. This means that they 
are respectively $\{1\}$ and $\{3\}$. Since ${\cal F}$ is a sigma algebra and 
$\{1, 3\}$, $F^{-1}(f(1))$, and $f^{-1}(f(3))$ are sets in ${\cal F}$, any intersection of these
sets is also in ${\cal F}$. But we have that $\{1\}$ and $\{3\}$ are sets that 
result from such intersections and consequently must be in ${\cal F}$. 
However, we know that they also are {\it not\/} in ${\cal F}$ -- contradiction. 
Therefore, our premise that $f(1)$ and $f(3)$ could 
take differing values is incorrect. Using the same argument one can show that
$f$ is constant on the other minimal set $\{2,4,6\}$.

Notice that while any function over the measure space $(P,X,{\cal E})$ is measurable, we 
can write down a specific function that is non-measurable with respect to ${\cal F}$.
We know that all we have to do is come up with a function that differs on either of the 
sets $\{1,3\}$ or $\{2,4,6\}$. For instance, the function:  $f(D_i) = i$, for $i\in\{1,2,3,4,5,6\}$, 
is a non-measurable function in ${\cal F}$.

\section{Conditional Expectation}
Given a probability space $(P,X,{\cal E})$, the conditional expectation of a measurable 
function $f$ with respect to a sub-sigma algebra ${\cal F}$ is the unique ${\cal F}$ 
measurable function (random variable) labeled, $\E{f/{\cal F}}$, such that 
$$
\eqalignno{
\int_{\Lambda} \E{f/{\cal F}} \, dP & 
= \int_{\Lambda} f \, dP \quad \forall \Lambda \in {\cal F} & (1) \cr
}
$$
That is, $\E{f/{\cal F}}$ is a measurable function in the probability space $(P,X,{\cal F})$.
Although it seems that $f$ itself satisfies this equation you have to be careful. The function 
we are looking for must be measurable with respect to ${\cal F}$, and since ${\cal F}$ is 
a sub-sigma algebra of ${\cal E}$, it is quite possible that $f$ is not ${\cal F}$ measurable.
However, if $f$ is measurable with respect to ${\cal F}$ then it is its own conditional 
expectation. We have the following basic facts about conditional expectation:

\beginEnum
\item{$\E{f/{\cal F}} = f$ when $f$ is measurable with respect to ${\cal F}$;}
\item{$\E{\E{f/{\cal F}}/{\cal F}} = \E{f/{\cal F}}$;}
\item{$\E{(f - \E{f/{\cal F}}) / {\cal F}} = 0$.}
\endEnum

Consider the function of the last section, $f(D_i) = i$, which is measurable in the 
space $(P,X,{\cal E})$. Let ${\cal F}$ be the sub-sigma algebra of the last section.
We now compute the conditional expectation of $f$ with respect to ${\cal F}$.
Using (1) we choose two $\Lambda$'s: $\Lambda_1 = \{1,3\}$ and $\Lambda_2 = \{2,4,6\}$. 
We have
$$
\eqalignno{
\int_{\Lambda_1} \E{f/{\cal F}} \, dP & 
= \int_{\Lambda_1} f \, dP & (2) \cr
}
$$
and
$$
\eqalignno{
\int_{\Lambda_2} \E{f/{\cal F}} \, dP & 
= \int_{\Lambda_2} f \, dP & (3) \cr
}
$$
We know $\E{f/{\cal F}}$ is constant on $\Lambda_1$. We label this value as
$C_{\Lambda_1}$. From (2) we have 
$$
\eqalignno{
\int_{\Lambda_1} \E{f/{\cal F}} \, dP  & =  \int_{\Lambda_1} f \, dP & \cr
C_{\Lambda_1} \int_{\Lambda_1} \, dP & = \int_{\Lambda_1} f \, dP & \cr 
C_{\Lambda_1} * P(\Lambda_1) & =  \int_{\Lambda_1} f \, dP = f(D_1) * P(D_1) + f(D_3) * P(D_3) & \cr 
C_{\Lambda_1} * P(\Lambda_1) & = 1 * \frac{1}{6} + 3 * \frac{1}{6}  =  \frac{2}{3} & \cr
}
$$
Since $P(\Lambda_1) = \frac{2}{6} = \frac{1}{3}$, we have that 
$$
C_{\Lambda_1} = 2
$$
Consequently,
$$
\E{f/{\cal F}}(D_1) = \E{f/{\cal F}}(D_3) = C_{\Lambda_1} = 2
$$
Likewise, $\E{f/{\cal F}}$ is constant on the set $\Lambda_2$. 
As we did above, we find the value of $\E{f/{\cal F}}$ on the set $\Lambda_2$.
Let $C_{\Lambda_2}$ be the constant value of $\E{f/{\cal F}}$ on $\Lambda_2$.
We have
$$
\eqalignno{
\int_{\Lambda_2} \E{f/{\cal F}} \, dP  & =  \int_{\Lambda_2} f \, dP & \cr
C_{\Lambda_2} \int_{\Lambda_2} \, dP & = \int_{\Lambda_2} f \, dP & (\dagger)\cr 
C_{\Lambda_2} * P(\Lambda_2) & =  \int_{\Lambda_2} f \, dP = f(D_2) * P(D_2) + f(D_4) * P(D_4) + f(D_6) * P(6) & \cr 
C_{\Lambda_2} * P(\Lambda_2) & = 2 * \frac{1}{6} +  4 * \frac{1}{6} + 6 \frac{1}{6} = 2 & \cr
}
$$
Since $P(\Lambda_2) = \frac{3}{6} = \frac{1}{2}$, we have that 
$$
C_{\Lambda_2} = 4
$$
Consequently,
$$
\E{f/{\cal F}}(D_2) = \E{f/{\cal F}}(D_4) = \E{f/{\cal F}}(D_6) = C_{\Lambda_2} = 4
$$
Since a function is determined once we know what its values are on every $x \in X$, 
we have found the conditional expectation function, $\E{f/{\cal F}}$, as we know its values on every $x \in X$.

In general, from equation $(\dagger)$ we see, that the value of the function $(f/{\cal F})$ evaluated on any `x` value in a minimal set, $\Lambda$,
is 

$$
\eqalignno{
	(f/{\cal F})(x) & = \frac{\int_\Lambda f \, dP}{\mu(\Lambda)} & \cr
}
$$

\bye


