\documentclass[12pt]{article}

\usepackage{amsthm, amsmath, amssymb}
\usepackage{amssymb}
\usepackage{tikz}

\setlength{\textwidth}{6.5in}
\setlength{\textheight}{8.0in}
\setlength{\hoffset}{-0.75in}
\setlength{\voffset}{-0.75in}

\setlength{\parindent}{0.0in}
\setlength{\parskip}{1.0\baselineskip}

\title{An Alternative Proof of an Elementary Property of Pythagorean Triples}
\author{R. Scott McIntire}
\date{Feb 5, 2023} 


\begin{document}

\maketitle

\section{Introduction}
When I was a freshmen in college I was briefly interested in the idea of 
using the areas of geometric figures to
produce interesting formulas. The example that inspired me was the way one 
may derive the Pythagorean formula
by examining a dissected square composed of 
an inscribed square and triangles. 

I tried to do something similar with a more complicated geometric figure and 
I came up with figure $(\ref{fig:geo})$.

By inscribing a circle%
\footnote{This can be achieved in the following way.
Take any point on the bisector of one of the angles formed
with the hypotenuse and one of the legs. By symmetry there is a unique circle
that touches the hypotenuse and the leg. It is clear that the radius 
of the circles grow and that the center of the circles gets closer to the
opposite leg as one moves out along the bisector.
Consequently, there is
a {\em unique\/} point on the bisector where the circle touches the opposite leg.
The coordinate of this point (see the figure) is (a-r, r). Using the fact 
(see the figure) that $(a-r) + (b-r) = c$ we see that regardless of 
which leg we choice for this procedure, the value of $r$ is the same as is
the center of the corresponding circle. Thus an inscribed circle exists
and is unique.}
in a {\em right\/} triangle, one can decompose the triangle 
into two pairs of right triangles and a square. The fact that the sub-triangles
are right triangles follows using the fact that 
any radial segment of a circle is perpendicular to its
associated tangent line. 

From the figure, the lower right four sided sub-figure has equal sides, $r$.
I claim that this sub-figure is a square and not a rhombus.
This can be seen as three of its interior angles are 
$90^\circ$; and, as the number of interior angles in a 4 sided {\em convex\/} 
polygon must sum to $360^\circ$, the last interior angle must also be $90^\circ$.
Consequently, the four sided figure is a square.

Below we find a well known relation between the sides, $a$, $b$, and $c$ of the
original triangle when the sides are integers in {\em reduced form\/}.%
\footnote{A triple of integers, $(a, b, c)$ is in reduced form if 
they have no common integer divisor(factor).}
This is done by investigating the fact that the sum of the areas
of these component triangles 
and square must be the same as the area of the original triangle. 

\section{A Basic Property of Primitive Pythagorean Triples}
Writing the equivalence of the area of the figure and its parts we have:

\begin{eqnarray}
    \overbrace{\vphantom{\frac{X^1}{X}}r^2}^{\text{Square}} 
    + \overbrace{2 \frac{r(a - r)}{2}}^{\text{Two lower triangles}} 
    + \overbrace{2 \frac{r(b-r)}{2}}^{\text{Two upper triangles}}  
    &=& \overbrace{\frac{a\, b}{2}}^{\text{Original triangle}}  
\end{eqnarray}

Simplifying, this is:
\begin{eqnarray}
r^2 + r(a - r) + r(b-r) = \frac{a\, b}{2} \label{area} 
\end{eqnarray}

One can solve the quadratic for $r$ or read off from the figure 
that $(a-r) + (b-r) = c$. Solving this last equation for $r$ gives:
\begin{eqnarray}
  r & = & \frac{a + b - c}{2} \label{radius}
\end{eqnarray}

We now consider primitive Pythagorean triangles -- right 
triangles with positive integer sides in reduced form.
In particular, we are interested in so-called primitive Pythagorean triples,
$(a,b,c)$ -- the lengths of the sides of primitive Pythagorean triangles.
Here $c$ is the length of the hypotenuse, while $a$ and $b$ are 
the lengths of the ``legs'' of the associated Pythagorean triangle.
\eject
What can we say about the evenness or oddness of the sides when they have no 
common factor? The possibilities are: 
\begin{itemize}
    \item{$c$ is even AND {\em exactly\/} one of the following is true:}
        \begin{itemize}
            \item{Both $a$ and $b$ are even;}
            \item{Both $a$ and $b$ are odd;}
            \item{One of either $a$ or $b$ is even and the other is odd.}
        \end{itemize}
    \item{$c$ is odd AND {\em exactly\/} one of the following is true:}
        \begin{itemize}
            \item{Both $a$ and $b$ are even;}
            \item{Both $a$ and $b$ are odd;}
            \item{One of either $a$ or $b$ is even and the other is odd.}
        \end{itemize}
\end{itemize}
Since the triple, $(a,b,c)$ is in reduced form we can eliminate the case 
of all three being even.
By the Pythagorean formula we know that $a^2 + b^2 = c^2$. Using this we can 
reduce the combinations further to the following:%
\footnote{We list the results of all combinations of even and odd numbers with 
respect to: addition, subtraction, and multiplication:
\begin{enumerate}
    \item{$E \pm E = E$; $ E \times E = E$;}
    \item{$E \pm O = O$; $ E \times O = E$;}
    \item{$O \pm O = E$; $ O \times O = O$.}
\end{enumerate}
These equations are to be read: An even integer plus (or minus) and even integer 
is an even integer.
Likewise, an even integer plus (or minus) an odd integer is an odd integer, etc.
These results are used throughout the rest of the document.
}
\begin{itemize}
  \item{$c$ is even AND both $a$ and $b$ are odd;}
  \item{$c$ is odd AND one of either $a$ or $b$ is even and the other is odd.}
\end{itemize}

Regardless of which choice occurs, equation $(\ref{radius})$ implies that 
the radius of the circle, $r$, is an integer.
\eject
Using this fact and equation $(\ref{area})$ we see that the quantity, 
$\frac{a \, b}{2}$, must be an integer; meaning, that the product 
of $a$, and $b$, $a \, b$ is even.
This knowledge eliminates the first choice; namely, that both $a$ and $b$ 
are odd. Consequently, we know:
\begin{itemize}
    \item{$c$ is odd AND one of either $a$ or $b$ is even and the other is odd.}
\end{itemize}
Without loss of generality we assume that $b$ is even; in which case, $a$ is odd.

Now, regardless of whether $r$ is even or odd, the left hand side of 
$(\ref{area})$ is even. This implies that $\frac{a \, b}{2}$
is even. Since $b$ is even, we can write this fraction as: $a \frac{b}{2}$.
So we know that $a \frac{b}{2}$ (the product of two integers) is even.
And we know that $a$ is odd; therefore, $\frac{b}{2}$ is even.
This means that $b$ is divisible by $4$.

In summary, we have deduced the following properties for primitive Pythagorean triangles
whose side lengths are in reduced form.
\begin{itemize}
  \item{One of the legs is odd; the other leg is even; and the hypotenuse 
      is odd.}
  \item{The even leg is divisible by $4$.}
  \item{The radius of the inscribed circle in the Pythagorean triangle
      is a positive integer.}
\end{itemize}

\begin{figure}[!ht]
\begin{center}
\begin{tikzpicture}[scale=0.75]
    % Draw the "a" leg.
    \draw (0,0)  --  (15,0) ;

    % Draw the "b" leg.
    \draw (15,0)  -- (15,20);

    % Draw the hypotenuse "c".
    \draw (0,0)  --  (15,20);

    % Label the bottom "a" leg.
    \node at (8,-1) {\Huge $a$};

    % Label the "b" leg.
    \node at (16,9) {\Huge $b$};

    % Label the hypotenus "c".
    \node at (5,9) {\Huge $c$};

    % Label the legs -- less the length of the square.
    \node at (17,12) {\Large $(b - r)$};
    \node at (4,-1) {\Large $(a - r)$} ;

    % The circle and its three "r" spokes.
    \draw (10,5)  circle (5);
    \draw (10,5) --  (10,0) ;
    \draw (10,5) --  (15,5) ;
    \draw (10,5) --  (6,8)  ;


    % Label the "r" spoke pointing at the hypotenuse "c".
    \node at (8,5.5) {\Large $r$};

    % Label the "r" spoke pointing to the "a" leg.
    \node at (9.5,3) {\Large $r$};

    % Label the "r" spoke pointing to the "b" leg.
    \node at (12,5.5) {\Large $r$};


    % Draw lines from the left and top vertices to the center of the circle.
    \draw (10,5) -- (0,0)  ;
    \draw (10,5) -- (15,20);


    % Label the edges of the square.
    \node at (12,-1) {\Large $r$};
    \node at (16,2) {\Large $r$} ;


    % Label the hypotenuses of the triangles that form the hypotenuse of the
    % original triangle, c.
    \node at (7,12) {\Large $(b - r)$};
    \node at (0,3) {\Large $(a - r)$} ;


    % Draw the two arcs and arc tics for the bottom right triangles.
    \draw (1,0) arc[start angle=0, end angle=53.14, radius=1] ; % The overall arc -- combines both arcs.
    \draw (0.8731, 0.2183) -- (1.1067, 0.2668)                ; % Arc tic for first arc.
    \draw (0.7200, 0.5400) -- (0.8800, 0.6600)                ; % Arc tic for second arc. 

    % Draw the two arcs and arc tics for the top right triangles.
    \draw (15,19) arc[start angle=270, end angle=233.12, radius=1]; % The overall arc -- combines both arcs.
    \draw (14.90  , 19.11  ) -- (14.88  , 18.91  )                ; % First arc tic for first arc. 
    \draw (14.8024, 19.1220) -- (14.7585, 18.9268)                ; % Second arc tic for first arc.

    % Tics for second arc.
    \draw (14.61  , 19.19  ) -- (14.53  , 19.01  ) ; % First arc tic for second arc.
    \draw (14.5271, 19.2343) -- (14.4220, 19.0641) ; % Second arc tic for second arc.


    % Draw a corresponding "90 degree angle" indicator box for the upper right corner of the square.
    \draw (14.5, 5.0)       -- (14.5, 4.5     ) ;
    \draw (14.5, 4.5)       -- (15.0, 4.5     ) ;

    % Draw a "90 degree angle" box indicating that the main triangle is a right triangle. 
    \draw (14.5,0)   -- (14.5,0.5) ;
    \draw (14.5,0.5) -- (15,0.5)   ;

    % Draw a "90 degree angle" box indicating the lower small triangle is a right triangle.
    \draw (9.5,0)   -- (9.5,0.5)   ;
    \draw (9.5,0.5) -- (10,0.5)    ;

    % Draw a "90 degree angle" box indicator for the lower left corner of the square.
    \draw (10.5,0)   -- (10.5,0.5)  ;
    \draw (10.5,0.5) -- (10,0.5)    ;

    % Draw a "90 degree angle" box indicating the lower left small triangle is a right triangle.
    \draw (6.3865, 7.6828)  -- (6.0865, 7.2828) ;
    \draw (6.0865, 7.2828)  -- (5.6865, 7.5828) ; 

    % Draw a "90 degree angle" box indicating the upper left small triangle is a right triangle.
    \draw (6.3865, 7.6828)  -- (6.6865, 8.0828) ;
    \draw (6.6865, 8.0828)  -- (6.2865, 8.3828) ;

    % Draw a "90 degree angle" box indicating the upper right small triangle is a right triangle.
    \draw (14.5, 5.0)       -- (14.5, 5.5     ) ;
    \draw (14.5, 5.5)       -- (15.0, 5.5     ) ;

\end{tikzpicture}
\end{center}
\caption{Dissected Right Triangle}
\label{fig:geo}
\end{figure}

\end{document}

