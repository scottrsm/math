\input macro


% - Define the macros needed to compute the cantor fractal.
\def\dup{%
\setbox0=\hbox{\copy0 \kern\wd0 \copy0}}

\newcount \fcount
\newcount \fcountsponge


\def\fracaux{
\ifnum\fcount>0
\advance\fcount by-1
\dup
\fracaux \else \fi
}

\def\cantor#1#2{%
\setbox0=\hbox{\vrule height1pt width #2pt}
\offinterlineskip
\fcount=#1
\fracaux
\vbox{\copy0}}


% - Define the macros needed to compute the 2-d example fractal.
\def\tri{%
\setbox0=\vbox{\hbox{\kern.5\wd0\copy0}\hbox{\copy0\copy0}}
}

\newcount \fcount

\def\fracrepstep{
\ifnum\fcountsponge>0
\advance\fcountsponge by-1
\tri
\fracrepstep
\else
\fi
}

\def\fsponge#1{%
\setbox0=\hbox{\vrule height1pt width1pt \kern1pt}
\offinterlineskip
\fcountsponge=#1
\fracrepstep
\vbox{\copy0}}



\mantitle{Defining Fractal Dimension}{R. Scott McIntire}{Mar 22, 2024}

\parindent=0pt
\parskip=.5\baselineskip
\baselineskip = 1.1\baselineskip

\subsection{Introduction}
{\parindent=0pt
In this paper we examine a notion of the size of a set called 
{\it fractal dimension} that enables one to more
finely distinguish the density of a given set than the usual
dimensional classification of $0, 1, 2, 3, \ldots$.%  
\ftn{The zero dimension of a set is the number of points in
that set.}
Such a notion is motivated by sets that, while constructed in one of the given
dimensions $0, 1, 2, 3, \dots$; and which are robust with respect to structure, 
nevertheless have zero "volume" in the ambinent space containing the set.
Such sets beg for a more nuanced "size" metric.

The sets that we are interested in are non-physical sets that
have the property that they retain structure no matter how much 
they are magnified. We call this class of sets
{\it fractal sets}, or
{\it fractals}. 

A common place example that resembles a fractal is the coast line of
a beach. If one were to take a map of the coast line and blow it up
again and again, one would find that it retains structure in the
sense that the expanded maps would not ``straighten out``.
Because this is a physical example, the complicated structure would fade
after a finite number of expansions. A true fractal, say representing a 
coast line, would retain complicated
structure no matter how many times a piece of it was magnified.
Because of this, the length of any section of this idealized coast line is
infinite. Another "natural" fractal is the surface of a idealized sponge.
While defined in a three dimensional world, one can construct such a sponge so that
it retains complicated structure no matter how many times it is magnified.
In addition, such a constructed sponge could have an infinite surface area and 
also have 0 volume. 

It would be convenient if there was a way to
measure the density of an idealized coast line to
indicate that it is, in some sense, larger than a one dimensional
object; or a way to show that the idealized surface of a sponge is larger than
two dimensions. This would allow us to compare the ``size'' of such sets
in a more precise way.

This is what the fractal dimension of a set tries to address. 
In the case of the coast line, it will give us a dimension between one and two
-- where the coast line ``naturally'' lives. The same with the surface of
the aforementioned sponge,
there will be a dimension between two and three where it naturally lives. In
each case, the fractal dimension is not an actual dimension, 
it is a way to quantify the density of such complex sets.
Yet, it should also "agree" with the usual dimensional size of an object 
if it is not a fractal. That is, the fractal dimension of a line segment should 
be 1; the fractal dimension of a triangle sitting in three dimensions should 
be 2, etc.

When studying any new phenomenon, it is natural to first examine the
simplest examples. The simplest such fractals are ones which have a
{\it self similarity} property; namely, they have the property that if
one magnifies a piece of the fractal by a certain amount, the original
fractal is recovered. This regular structure makes them easier to
study mathematically than a general fractal. However, before we start
with more specific examples of fractals, we need to understand what 
one might mean by fractal dimension.

\subsection{Volume of an n-Dimensional Sphere}
The n-dimensional volume of any sphere is easy to compute if one 
knows its surface area. Since the surface area of an n-dimensional sphere 
with radius, $r$, is $r^{n-1}$ times the surface of the unit sphere, 
it suffices to know the surface area of the unit sphere.
In what follows we will refer to an n-dimensional sphere of radius, $r$, 
as $B_n(r)$, and $B_n$ as the specific n-dimensional sphere of radius $1$.

The relationship between the volume of an n-dimensional sphere, $B_n(r)$ 
and the surface area of $B_n$ follows from the following formula:%
\ftn{We use the functions: ${S}_n$ and ${V}_n$ 
to which compute the surface area and volume of n-dimensional spheres.} 

${V}_n(r) =  \int_{\partial B_n} \int_0^r \tau^{n-1} \,dS \,d\tau = \left(\int_{\partial B_n} \, dS\right) \left( \int_0^r \tau^{n-1} \, d\tau \right)$

The iterated integrals on the right hand side of the above equation can 
be easily computed, yielding: ${V}_n(r) = {S}_n(1) \frac{r^n}{n}$.

What remains now is to find a formula for the surface area of $B_n.$
This can be done by expressing an easily computable 
n-dimensional integral in $R^n$ as $n$ iterated integrals. 
And more importantly, this integral may be represented as a {\it different\/} 
integral involving the surface are of $B_n$.

$$
\int_{-\infty}^{\infty} e^{-(x_1^2 + x_2^2 + \cdots + x_n^2)} \, dx_1\, dx_2
\, \cdots \, dx_n = \int_{\partial B_n} \int_0^\infty e^{-r^2} r^{n-1} \,
dS\, dr
$$
The left side of the above can be written as an iterated integral of identical 
one dimensional integrals. The right side of the equation is the same 
integral in ``spherical coordinates``: this is a convenient choice as 
the integrand is spherically symmetric. The left side simplifies to 
$\int_{-\infty}^{\infty} e^{-x^2} \, dx$ raised to the $n$th power;
which is $\pi^{n/2}$. Substituting $t = r^2$ in the inner integral on
the right hand side we have:

$$
\pi^{n/2} = \frac{1}{2} \int_{\partial B_n} \int_0^\infty 
e^{-t} t^{\frac{n}{2}-1} \,dt \, dS
$$
which may be written:%
\ftn{The function $\Gamma(n)$ is called the Gamma function and 
has the property: $\forall n\in Z^+, \Gamma(n+1) = n \Gamma(n)$.}
$$
2 \pi^{n/2} = {S}_n(1) \Gamma(n/2)
$$
Or,
$$
{S}_n(1) = \frac{2 \pi^{n/2}}{\Gamma(n/2)}
$$
Consequently,
$$
{S}_n(r) = \frac{2 \pi^{n/2}}{\Gamma(n/2)}r^{n-1}
$$
And so the volume of an n-dimensional sphere, $B_n(r)$, of radius $r$ is:%
\ftn{The volume of an n-dimensional sphere scales by the radius to 
the $n^{\rm th}$ power.}
$$
{V}_n(r) = \int_{B_n} \, d\mu = \int_{\partial B_n} 
\int_0^r \tau^{n-1} \, d\tau \, dS = \left(\frac{r^n}{n}\right){S}_n(1)
$$
Which is
$$
{V}_n(r) = \frac{2\pi^{n/2}}{n \, \Gamma(n/2)} r^n
$$

\subsection{Fractal Dimension and Associated Volume of a Set}
Although we derived the formula for the n-dimensional volume of a sphere.
The formula may take non-integral values. 
Armed with this formula, we make
the following definition.

\proclaim Definition. For any $\alpha \in R^+$, the $\alpha$-dimensional 
volume, $V_\alpha$, of an n-dimensional sphere 
of radius r is defined to be:%
\ftn{The set $R^+$ is defined 
by: $R^+ = \{x \, | \, x\in R \; \wedge \; x\ge 0\}$.}
$$
{V}_\alpha(r) = \frac{2\pi^{\alpha/2}}{\alpha \, \Gamma(\alpha/2)} r^\alpha
$$
This is a mathematical construct, not a physical one; there is no claim
here of a physical fractional dimension. Also, note that the value is 
independent of the ambient space of the ball, $R^n$. That is, $\alpha$ 
can be either larger or smaller than $n$.

Given that we have defined the fractional dimensional volume of a sphere 
in "dimension" $\alpha$ we wish to extend this to a definition of the 
fractional dimensional volume of {\it any\/} set.

\proclaim Definition. The $\alpha$-dimensional ($\alpha \in R^{+}$) {\it covering\/} 
volume of a set $S \in R^n$, ${\cal V}_{\alpha, \epsilon}(S)$,
with parameter, $\epsilon \in R^+$, is defined as\ftn{The diameter of a 
set is defined by: 
{\rm Diam}(S) = $\sup\{|x - y| : x,y \in S\}$.}
$$
\eqalign{
	{\cal V}_{\alpha, \epsilon}(S) & = \inf_{\{O_i\}_{i=1}^N} 
    \left\{ \sum_{i=1}^N {V}_\alpha\left({\rm Diam}(O_i) / 2\right) \
    : S \subseteq \bigcup_{i=1}^N\limits O_i \; \wedge \; \left(\mathop{\rm Max}\limits_{i \in [1,N]} 
    {\rm  Diam}(O_i)\right) \ge \epsilon \right\} 
}
$$

Roughly speaking, for any given finite covering, 
$\left\{ O_i \right\}_{i=1}^N$, and value $\epsilon$, 
we estimate the $\alpha$-dimensional volume of the set, $S$, by summing 
up the $\alpha$-dimensional volumes of the collection of spheres, 
$\left\{ P_i \right\}_{i=1}^N$, where each $P_i$ is the 
smallest sphere containing $O_i$. In this case the radius of a given 
$P_i$ is ${\rm Diam}(O_i) / 2$.
The $\alpha$ covering volume of a set, $S$, is the infimum over all 
such covers. The only restriction of the covering sets is that they are 
finite and the corresponding spheres have radius no smaller than $\epsilon$.

The intent of this definition is to give a practicable upper estimate of 
the eventual $\alpha$ volume of a set. We would also like that the above 
would give finer and finer estimates of the $\alpha$ volume with 
smaller $\epsilon$ yielding smaller estimates.

This is indeed the case:
$$
\eqalign{
	{\cal V}_{\alpha, \epsilon_1}(S) & \le {\cal V}_{\alpha, \epsilon_2}(S) \quad \quad \hbox{whenever  } \epsilon_1 \le \epsilon_2
}
$$
This follows as the set of coverings used to compute 
${\cal V}_{\alpha, \epsilon_1}(S)$ includes the set of all coverings used 
to compute ${\cal V}_{\alpha, \epsilon_2}(S)$. Therefore, the 
infimum over all coverings associated with $\epsilon_1$ cannot be larger
than the infimum associated with $\epsilon_2$.

Using the covering volume, we may define the fractional dimensional volume by
taking a limit of increasing refined covers.

\proclaim Definition. The $\alpha$-dimensional ($\alpha \in R^{+}$)
volume of a set $S \in R^n$, ${\cal V}_\alpha(S)$,
is defined as\ftn{%
This definition is well defined, as the limit of a monotonic sequence exists,
provided we acknowledge that it may be infinite.}
$$
\eqalign{
	{\cal V}_\alpha(S) &= \lim_{\epsilon \rightarrow 0^+} {\cal V}_{\alpha, \epsilon}(S)
}
$$
It is not immediately clear how the $\alpha$ volumes of a set behave as the 
dimension changes. It turns out that these volumes are non-increasing 
with increasing $\alpha$. This follows from the following two facts:

\beginEnum
\enum{For $r \le 0.1$, the $\alpha$-dimensional volume, $V_{\alpha}(r)$, of 
a sphere monotonically decreases with $\alpha$.}
\enum{This means that for a given bounded set $S$ and for any fixed 
		$\epsilon \le 0.1$, ${\cal V}_{\alpha, \epsilon}(S)$ is 
		non-increasing in $\alpha$.}
		\enum{If for some $\alpha_0 \in R^+$, ${\cal V}_{\alpha_0}(S) < \infty$, then
		${\cal V}_{\alpha}(S)$ is monotonically decreasing for 
		$\alpha \ge \alpha_0$.}
\endEnum

While these statements define the $\alpha$ volume for any given set, $S$,
they do not tell us the ``natural'' fractal volume of $S$.
The following definition does precisely that.

\proclaim Definition. The {\bf fractal dimension} of a 
set $S \in R^m$ is defined as
$\alpha^* = \inf\{\alpha : {\cal V}_\alpha(S) = 0 \}$; the associated 
{\bf fractal volume} of $S$ is defined to be ${\cal V}_{\alpha^*}(S)$.

Roughly speaking, this says that the fractal dimension of a set $S$ is the
``smallest'' $\alpha$ such that the $\alpha$-dimensional volume of $S$ 
is non-zero.

Note that the fractal dimension of a set, $S$, does not depend on 
the ambient space $S$ sits in.

\subsection{Computing the Fractal Dimension of a Simple Set}

The fractal dimension of a set $S$ should produce the usual
answer when $S$ is a non-fractal set. We show this for the case of a
straight line.

\example{Example}{A line segment in two dimensions.}
We wish to find the fractal dimension and fractal volume of the line
segment 
$$
L = \{t{\bf x} + (1-t){\bf y} : t \in [0,1] , {\bf x, y}~\in~R^2\}
$$
While this line segment sits in the two dimensional plane, we wish 
to find its ``natural'' dimension and associated volume.

For a given $n$, consider the sets $\{O_{n,i}\}_{i=0}^{n-1}$, with $O_{n,i}~=~
[\frac{i}{n}{\bf x} + (1-\frac{i}{n}){\bf y}, \; \frac{i+1}{n}{\bf x} + (1 -
\frac{i+1}{n}){\bf y}]$. The collection of line segments, $O_{n,i}$, act
as a cover for $L$; that is, $L =
\bigcup_{i=0}^{n-1}\limits O_{n,i}$. For a given $\alpha$ the
covering volume of $L$ with $\epsilon = \frac{1}{n} \|{\bf x} - {\bf y}\|$ 
is bounded above by
$$
\sum_{i=0}^{n-1} {\rm V}_\alpha({\rm Diam}(O_{n,i})/2) = \sum_{i=0}^{n-1}
{\rm V}_\alpha(\|{\bf x} - {\bf y}\|/2n) =  \sum_{i=0}^{n-1} \frac{2
\pi^{\alpha / 2}}{\alpha \Gamma(\alpha / 2)} \frac{\|{\bf x} - {\bf
y}\|^\alpha}{(2n)^\alpha} = \frac{2 \pi^{\alpha / 2} \|{\bf x} - {\bf
y}\|^\alpha}{2^\alpha \alpha \Gamma(\alpha / 2)} \frac{n}{n^\alpha}
$$

As $n$ increases, the covering becomes more refined and 
the $\alpha$-dimensional volume is dominated by the term
$\frac{n}{n^\alpha}$. Clearly, this term tends to zero as n tends to
infinity for $\alpha > 1$. This implies ${\cal V}_\alpha(L) = 0$ 
for $\alpha > 1$. The smallest $\alpha$ that yields a non-zero 
value is $\alpha = 1$. In this case 
${\cal V}_1(L) = \left. \frac{2\pi^{\alpha / 2} \|{\bf x} - {\bf
y}\|^\alpha}{2^\alpha \alpha \Gamma(\alpha / 2)} \right|_{\alpha=1} = \| {\bf
x} - {\bf y}\|$. 
Therefore, the fractal dimension of the line segment is
1 and its fractal volume is the length of the line segment, $\|{\bf x} - {\bf
y}\|$. This, of course, coincides with what we would previously regard
as the dimension and ``volume'' of a line segment.

Note that the answer would stay the same if the line segment was 
embedded in $R^n$ (for any $n \in {\cal Z}^+$) rather than $R^2$.


This result is true for any ``ordinary'' set: its fractal dimension is its
``ordinary'' dimension. This means that we can think of fractal
dimension as a {\it generalization} of dimension. It is 
instructive to repeat the above analysis and compute the fractal dimension 
and fractal volume of a simple fractal.


\subsection{The Cantor Set: A Simple Fractal Set}

As mentioned at the beginning of this paper, the simplest fractals to
study are the ones with regular behavior; in particular, fractals with a
self similarity property are more mathematically tractable. Perhaps
the simplest fractal of this type is 
the Cantor set. It is a self-similar set formed by doing the following:
Take the unit interval, [0,1], and remove the middle third, (1/3,
2/3). This leaves the intervals, [0,1/3] and [2/3,1]. Now remove the
middle third of each of these two intervals, this gives four
intervals.
If we repeat this process indefinitely with the remaining
intervals, the resulting set is called the Cantor set, C. See 
the appendix for a graphical depiction of the construction process . 
The Cantor
set has the property that $3 * \left([0,1/3] \cap C\right) = C$; that is, 
it is a self-similar fractal. 

The one dimensional length of the Cantor set is
easy to compute, it must be one (the length of the interval $[0,1]$) 
less the sum of all the lengths of the
intervals removed. This is computed as $1
- \sum_{n=1}^\infty (2^{n-1}/3^n) = 1 - \frac{1}{2}
\sum_{n=1}^\infty (2/3)^n = 0$. Since the Cantor set has zero length,
perhaps its zero dimensional ``length'' is a more natural description
of its size. But its zero dimensional length is equal to the 
number of points in the Cantor set; and, it turns out that this is infinite. 
What's more startling is
that, in a sense that can be made precise, there are as many points in
the Cantor set as there are in the interval $[0,1]$. So, although the 
Cantor set has no length, it is comparable in complexity
to $[0,1]$. This suggests that it might have a natural dimension
between zero and one.

\subsection{The Fractal Dimension of the Cantor Set and its Volume}
We compute the fractal dimension of the Cantor set. To do so, 
consider a refinement of coverings of the Cantor set, $C_n$, where
$C_n$ is the union of the set of $2^n$ intervals that remain after 
the $n^{th}$ stage of the Cantor construction process. Each of these 
intervals has length $(1/3)^n$. 
Clearly, $C \subseteq C_n$, and $C = \bigcap_{n=1}^\infty\limits C_n$. 

We label $I_{n,i}$ as the $i^{th}$ interval in the set
$C_n$. Each of these intervals has a diameter, $\epsilon_n = \left(\frac{1}{3}\right)^n$.
The collection, $\bigcup_{i=1}^{2^n}\limits I_{n,i}$, provides an
approximation to ${\cal V}_{\alpha, \epsilon_n}(C)$:


$$
{\cal V}_{\alpha, \epsilon_n}(C) \le \sum_{i=1}^{2^n} {\rm V}_\alpha({\rm
  Diam}(I_{n,i}) / 2)
$$

As ${\rm Diam}(I_{n,i}) = \left(\frac{1}{3}\right)^n$, this becomes


$$
{\cal V}_{\alpha, \epsilon_n}(C) \le \sum_{i=1}^{2^n} \frac{2
  \pi^{\alpha/2}}{ \alpha \Gamma(\alpha/2) 2^\alpha 3^{n\alpha}} 
  \approx  \frac{
  2^{1 - \alpha} \pi^{\alpha/2}}{\alpha \Gamma(\alpha/2)} \;
  \sum_{i=1}^{2^n}  
\frac{1}{3^{n\alpha}}
$$

Or, 
$$
{\cal V}_{\alpha, \epsilon_n}(C) \le \frac{ 2^{1-\alpha} \pi^{\alpha/2}}{\alpha
  \Gamma(\alpha/2)} \left(\frac{2}{3^\alpha}\right)^n
$$


If $\frac{2}{3^\alpha} < 1$ then, in the limit as 
$n \rightarrow \infty$ 
this limit will be zero. Certainly,
this is the case for $\alpha \ge 1$; in which case 
$$
{\cal V}_\alpha(C) =  \lim_{n \rightarrow \infty} {\cal V}_{\alpha, \epsilon_n}(C) \le \lim_{n \rightarrow \infty}  \frac{ 2^{1-\alpha} 
\pi^{\alpha/2}}{\alpha
  \Gamma(\alpha/2)} \left(\frac{2}{3^\alpha}\right)^n = 0 
$$

The smallest value of $\alpha$ that gives a non-zero ${\cal V}_\alpha(C)$ 
is $\alpha$ such that $\frac{2}{3^\alpha} = 1$. 
This occurs when
$\alpha^* = \ln 2 / \ln 3$. Therefore, the fractal dimension of the
Cantor set is $\alpha^* = \ln 2 / \ln 3 \approx 0.63093$. 
Its fractal volume in this
dimension is ${\cal V}_{\alpha^*}(C) =  \frac{2^{1-\alpha^*} 
\pi^{\alpha^*/2}}{\alpha^*
\Gamma(\alpha^* / 2)} \approx 1.03505 $.

\subsection{Conclusion}
We have shown that there are sets with sufficient complexity that the
usual categorization of dimension as a measure of size is lacking. 
We have further shown
that the notion of fractal dimension, as an extension to the usual
notion of dimension, provides a more discriminating view of the size
of a set.

While physical examples like a coast line or the surface of a sponge do 
not really have
complexity on an infinite scale, we idealize them as such with
fractals. Just as ``dimensionless'' points are an idealization useful
to describe space, fractals are also a useful idealization. 
As with all such mathematical abstractions, they 
allows us to view and compute what they model in a simpler way.

\subsection{Appendix A: Examples of Self-Similar Fractals \hfill}
We briefly describe two fractal sets. 
The first set lives in 1 dimension, but its fractal dimension is between $0$ and $1$\dots
The set is called the Cantor set, named after the mathematician Georg Cantor.
This set may be created by creating intermediary sets $C_0, C_1, \ldots$ 
and setting the Cantor set, $C$, to be the intersection of these sets:
$$
\eqalign{
    C &= \bigcap_{n=0}^\infty\limits C_n
}
$$
The intermediate sets have the property that $\forall n, C_{n+1} \subseteq C_n$.
These sets are defined by:
$$
\eqalign{
    C_0 &= [0, 1] \cr
    C_{n+1} &=  \hbox{For each contiguous segment of the set $C_n$, remove the open interval that is its middle third.} \cr
}
$$

As the sets, $C_n$, are nested, each successive set is an improved approximation to the Cantor set.
The first 6 intermediate sets are graphed below:

\vfil\break
{
$C_0$:\cantor{0}{360}
}
{
\vskip .1in
{
$C_1$:$\!\!$\cantor{1}{120}
}
\vskip .1in
{
$C_2$:$\!\!\!\!$\cantor{2}{40}
}
\vskip .1in
{
$C_3$:$\!\!\!\!\!\!$\cantor{3}{13.33333}
}
\vskip .1in
{
$C_4$:$\!\!\!\!\!\!\!\!$\cantor{4}{4.4444}
}
\vskip .1in
{
$C_5$:$\!\!\!\!\!\!\!\!\!\!$\cantor{5}{1.48148}
}
\vskip .1in
}

The second set is another self-similar set which lives in 2 dimensions while its 
fractal dimension is between $1$ and $2$. The set is a scaled version of the 
lower triangular portion of itself.

Similar to the graphical display of the Cantor set construction, below we show 
a graph of 8 ``levels'' of its construction without going into construction details.

\vskip .2in
{
\fsponge{8}
}

\bye
