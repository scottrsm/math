\documentclass{article}

\usepackage{amsmath}


\setlength{\textwidth}{6.5in}
\setlength{\textheight}{7.5in}
\setlength{\hoffset}{-0.75in}
\setlength{\voffset}{-0.75in}
\setlength{\parindent}{0pt}
\setlength{\parskip}{0.5\baselineskip}

\title{Derivation of the Exponential Moving Average}
\author{R. Scott McIntire}
\date{Aug 22, 2023}

\begin{document}

\maketitle


\section{Overview}

The essence of the {\em exponential moving average\/} (EMA) is to provide 
a {\em localized\/} estimate of a time series as it proceeds. It does 
this by using a weighted average of the values in a "window" around each point 
of the series in way that each summand in 
the average is multiplied by powers of a factor, $\lambda$,
in the interval $(0,1)$.
In what follows below we will assume a 1-based index, and that 
the data series ${\bf x}$ has length $N$.

Specifically, the exponential moving average (over a window of length $m$) 
is defined as:
\begin{eqnarray}
    e_n & = & \sum_{k=0}^{m-1} x_{n-k} w_{k+1} \label{ema_def}
\end{eqnarray}
To get the {\em exponential decay\/} in weights we assume that $w_{k+1} = \lambda w_k$.%
\footnote{The reason this is called exponential is that it is the discrete
analog of continuous exponential decay. To see this let $f(x) = e^{-kt}$ 
for $k > 0$. Then sampling this function at regular intervals of say $1$ 
we get the sequence: $\{ e^{-k}, e^{-2k}, e^{-3k}, \ldots \}$.
Here we can see that the elements of the sequence are powers of a value, $\lambda = e^{-k}$.
That is, the sequence is nothing other than: $\{\lambda, \lambda^2, \lambda^3, \ldots \}$.}
Since the parameter $\lambda$ is in the interval $(0,1)$, it represents a decay factor. 
Notice that the indices of ${\bf x}$ and that of ${\bf w}$ go in the "opposite direction".
This is done so that the largest weights are associated with the most recent data
in the averaging window.

Before we go further, there is a problem with the formula as stated.
If $n = 1$; or, in fact any value that is less than or equal to $m$,
the expression $x_{n-k}$ does not make sense for $k \ge n$. That is, at the 
start of the computation and for a little while, there is not a full window
of values over a window of length $m$. As there is no information before $x_1$,
we interpret these values to be $x_1$. 
We will apply this principal more generally below. 
That is, if ${\bf z}$ is a data series,
we extend the definition of ${\bf z}$ so that it is defined for
indices less than 1 as:

\begin{eqnarray}
 z_{k} & = & \begin{cases}
     z_{k} & k \ge 1; \cr
     z_1 & \text{otherwise}. \cr
    \end{cases}
\end{eqnarray}

While we can use $(\ref{ema_def})$ to compute the EMA it is {\em inefficient\/}. 
The purpose of this
paper is to derive a recursive formula that is fast to compute.
The formula takes advantage of the fact that when the weights are exponentials
(powers of $\lambda$) we don't have to recompute the bulk of the sums in the 
averaging window.

Regardless of the averaging scheme, we want our weights to sum to one.%
\footnote{If we consider the underlying data to be generated from a series of identically distributed random
variables; then, treating the series elements as random variables we can view 
the moving average as an estimate of their mean, $\mu$. 
To be a good estimate we which this estimate to be {\em unbiased\/}:
$E\left[\sum_{k=1}^m x_{n-k} w_{k+1}\right] = \sum_{k=0}^{m-1} E[x_{n-k}]w_{k+1}$.
Which is $\mu \sum_{k=0}^{m-1} w_{k+1} = \mu \sum_{k=1}^{m} w_k = \mu$.}
But we also want the weights to be exponential; that is, $w_{k+1} = \lambda w_{k}$.

Ideally, all we want to do is give the decay factor, $\lambda$. If so, 
how do we determine the weights?
The weights can be constructed as follows once a $\lambda$ is given.

For $i \in [1, m]$ define
\begin{eqnarray}
    {\hat w}_i & = & \lambda^i 
\end{eqnarray}
The ${\hat {\bf w}}$ have the property that they decay with $\lambda$.
That is, $\lambda w_{k} = w_{k+1}$. However, how do we ensure that
these values sum to 1 over the window? 

To do this, we simply normalize the weights:
\begin{eqnarray}
w_i & = &  \frac{{\hat w}_i}{\sum_{k=1}^m {\hat w}_k} \quad \forall i \in [1, m]
\end{eqnarray}
In doing this, did the proposed weights, ${\bf w}$, lose the decay factor
property? The answer is no, the property remains: $\lambda w_{i} = w_{i+1} \quad \forall i \in [1, m-1]$.

{\bf Note:\/} We will use this fact in our derivation below.

\section{Recursive Formula for EMA}
Define the exponential moving average of $x$ as:
\begin{eqnarray}
e_n & = & \sum_{k=0}^{m-1} x_{n-k} w_{k+1}
\end{eqnarray}
We examine the formula for the next element in the smoothing sequence 
and try to relate it to the present element at index $n$.

\begin{eqnarray*}
e_{n+1} & = & \sum_{k=0}^{m-1} x_{n+1-k} w_{k+1} \\
        & = & \sum_{k=0}^{m-1} x_{n-(k-1)} w_{k+1} \\
        & = & \sum_{k=-1}^{m-2} x_{n-k} w_{k+2}  \\
        & = & \sum_{k=0}^{m-2} x_{n-k} w_{k+2}  + x_{n+1} w_1 \\
        & = & \lambda \sum_{k=0}^{m-2} x_{n-k} w_{k+1}  + x_{n+1} w_1  \quad \text{(Since $\lambda w_{k+1} = w_{k+2}$)} \\
        & = & \lambda \sum_{k=0}^{m-1} x_{n-k} w_{k+1}  + x_{n+1} w_1  
                - \lambda x_{n-(m-1)} w_m  \\
        & = & \lambda \sum_{k=0}^{m-1} x_{n-k} w_{k+1}  + x_{n+1} w_1  
                - \lambda x_{(n+1)-m} w_m 
\end{eqnarray*}

Finally, we may write
\begin{eqnarray}
    e_{n+1} & = & \lambda e_n + x_{n+1} w_1 - \lambda x_{(n+1)-m} w_m \quad \forall n \in [1, N-1] \\
e_1     & = & x_1
\end{eqnarray}

\section{Recursive Formula for the Std of EMA}
In practice, one also wants to know a localized standard deviation of the process.
In applications, one can provide "Bollinger" bands around the EMA estimate.
These bands are some multiple of the standard deviation of the process.

We derive the standard deviation of the exponential moving average starting
with the definition of the exponential moving average and it moving variation.%
\footnote{This variance formula needs to be corrected by a factor which we 
leave to the end. The correction is a factor that is used to make the 
estimated standard deviation of the moving average {\em unbiased\/}.
The formula for this factor is independent of the structure of the weights.
This means that it can be pre-computed independently of the data series.}

\begin{eqnarray}
e_n & = & \sum_{k=0}^{m-1} x_{n-k} w_{k+1} \\
v_n & = & \sum_{k=0}^{m-1} (x_{n-k} - e_{n-k})^2  w_{k+1}
\end{eqnarray}

As in the previous section, we look at the next element of the sequence 
and try to show its relation to the present value at index $n$.

\begin{eqnarray*}
v_{n+1} & = & \sum_{k=0}^{m-1} (x_{(n+1)-k} - e_{(n+1)-k})^2  w_{k+1} \\
        & = & \sum_{k=0}^{m-1} (x_{n-(k-1)} - e_{n-(k-1)})^2  w_{k+1} \\
        & = & \sum_{k=-1}^{m-2} (x_{n-k} - e_{n-k})^2  w_{k+2} \\
        & = & \sum_{k=0}^{m-2} (x_{n-k} - e_{n-k})^2  w_{k+2}  + ( x_{n+1} - e_{n+1} )^2 w_1 \\
        & = & \lambda \sum_{k=0}^{m-2} (x_{n-k} - e_{n-k})^2  w_{k+1}  + ( x_{n+1} - e_{n+1} )^2 w_1  \quad \text{(Since $\lambda w_{k+1} = w_{k+2}$)} \\
        & = & \lambda \left[ \sum_{k=0}^{m-1} (x_{n-k} - e_{n-k})^2  w_{k+1} - ( x_{(n+1)-m} - e_{(n+1)-m} )^2 w_m \right]
            + ( x_{n+1} - e_{n+1} )^2 w_1 
\end{eqnarray*}
Therefore,

\begin{eqnarray}
    v_{n+1} & = & \lambda v_n + ( x_{n+1} - e_{n+1} )^2 w_1 - \lambda ( x_{(n+1)-m} - e_{(n+1)-m} )^2 w_m \quad \forall n \in [1, N-1] \\
    v_1  & = & w_1 (x_1 - e_1)^2 = 0
\end{eqnarray}

Finally, the standard deviation of the exponential moving average is:
\begin{eqnarray}
    s_{n} = \sqrt{\frac{v_n}{1 - \sum_{i=1}^m w_i^2}} 
\end{eqnarray}
The variance corrective factor, $1 - \sum_{i=1}^m w_i^2$ is a 
one time calculation independent of $n$. This factor is the standard 
correction when determining empirical variance of a weighted average.

{\bf Note:\/.} In practice, one may wish to use some prior estimate of the standard deviation of the process
and smoothly move from that to the estimate above over the averaging windows when $n \in [1, m]$.


\section{How to Specify the $\lambda$ Factor}
When using the EMA, practitioners do not specify the factor $\lambda$ directly; 
instead they provide the "half-life" of the decay.
By the half-file of $\lambda$, we mean the number of powers of $\lambda$, $N$, so that
$\lambda^N \approx \frac{1}{2}$. 

In our case we are interested in the reverse; namely, given a half life, $N$, 
determine the factor $\lambda$. To do this we solve the equation $\lambda^N = \frac{1}{2}$
for $\lambda$.
Taking the $\log$ of this equation and solving for $\lambda$ we have:%
\footnote{The log used below is the logarithm base $e$.}

\begin{eqnarray*}
    \log(\lambda^N) & = & \log(\frac{1}{2}) \\
    N \log(\lambda) & = & \log(1) - \log(2) \\
    \log(\lambda) & = & -\frac{\log(2)}{N}
\end{eqnarray*}
The formula to determine $\lambda$ given the half-life $N$ becomes:

\begin{eqnarray}
    \lambda = e^{-\frac{\log(2)}{N}}
\end{eqnarray}

\end{document}


