\input epsf
%\input eplain

%\voffset=.5in


% Bold and italic bold scaled fonts for sections and examples.
\font\small = cmr7
%\font\smallmath = cmu10
\font\sxmplbx = cmbx10 
\font\sxmplbxti = cmbxti10 
\font\xmplbx = cmbx12 % scaled \magstephalf
%\font\xmplbxti = cmbxti10 scaled \magstephalf
\font\mxmplbx = cmbx12 at 16pt
\font\bxmplbx = cmti12 at 18pt %logo1%0cmbx12 scaled \magstep2

% A title macro. First parameter is the title of the article.
% My name and the current time appear in the title section.
\def\title#1{%
{\bxmplbx {\centerline {#1}}}
\vskip .2in
  {\leftskip = 2.5in%
  \par\noindent\llap{\hbox to 2.5in{\hfil {\bf Author:} }}{\it R. Scott
  McIntire}\par\noindent {\it Financial Engineering}%
  }%
\vskip .1in
 {\leftskip = 2.5in%
  \par\noindent\llap{\hbox to 2.5in{\hfil {\bf Date:} }} \date \par%
  }%
\vskip .25in
\hrule height 1pt
\vskip .1in
}

% A section header. 
\def\section#1{\vskip .5in \centerline{\mxmplbx #1} \bigskip}

% A subsection header. 
\def\subsection#1{\vskip .25 in \leftline{\xmplbx #1} \medskip}

% Chapter macro. If at first chapter set \pageno to 1.
\count3=1
\def\chapter#1{%
\vfil
\supereject
{\bxmplbx {\centerline {Chapter \number\count3: #1}}}
\vskip .2in
\hrule height 1pt
\vskip .2in
\ifnum \count3 = 1 \pageno=1 \fi
\global\advance\count3 by 1
}%

% Appendix macro.
\def\appendix#1{%
\vfil
\supereject
\section{#1}
}%

% A title macro. First parameter is the title of the article.
% My name and the current time appear in the title section.
\def\titledate#1#2{%
{\bxmplbx {\centerline {#1}}}
\vskip .2in
  {\leftskip = 2.5in%
  \par\noindent\llap{\hbox to 2.5in{\hfil {\bf Author:} }}{\it R. Scott
  McIntire}\par\noindent {\it Financial Engineering}%
  }%
\vskip .1in
 {\leftskip = 2.5in%
  \par\noindent\llap{\hbox to 2.5in{\hfil {\bf Date:} }}#2 \par%
  }%
\vskip .25in
\hrule height 1pt
\vskip .1in
}


% A title macro. First parameter is the title of the article.
% Second parameter is who; third parameter is date.
\def\mantitle#1#2#3{%
{\bxmplbx {\centerline {#1}}}
\vskip .2in
  {\leftskip = 2.5in%
  \par\noindent\llap{\hbox to 2.5in{\hfil {\bf Author:} }}{\it #2}
  \par%
  }%
\vskip .1in
 {\leftskip = 2.5in%
  \par\noindent\llap{\hbox to 2.5in{\hfil {\bf Date:} }}#3\par%
  }%
\vskip .25in
\hrule height 1pt
\vskip .1in
}

% A title macro. First parameter is the title of the article.
% Second parameter is who; third parameter is date.
\def\mansubtitle#1#2#3#4{%
{\bxmplbx {\centerline {#1}} 
          {\centerline {#2}}}
\vskip .2in
  {\leftskip = 2.5in%
  \par\noindent\llap{\hbox to 2.5in{\hfil {\bf Author:} }}{\it #3}
  \par%
  }%
\vskip .1in
 {\leftskip = 2.5in%
  \par\noindent\llap{\hbox to 2.5in{\hfil {\bf Date:} }}#4\par%
  }%
\vskip .25in
\hrule height 1pt
\vskip .1in
}


% Example macro - Example #1: #2.
\def\example#1#2{\medskip \leftline{\sxmplbx #1: \sxmplbxti #2} \medskip}

% Produce a parenthesized matrix of variable #1 of size #2 by #2.
\def\mat#1#2{%
  \pmatrix{%
    #1_{11} & #1_{12} & \ldots & #1_{1#2} \cr
    #1_{21} & #1_{22} & \ldots & #1_{2#2} \cr
    \vdots & \vdots & \ddots & \vdots \cr
    #1_{#21} & #1_{#22} & \ldots & #1_{#2#2} \cr}}

% Produce a diagonal parenthesized matrix of #1 of size #2 by #2.
\def\diag#1#2{%
  \pmatrix{%
    #2 & 0 & \ldots & 0 \cr
    0 & #2 & \ldots & 0 \cr
    \vdots & \vdots & \ddots & \vdots \cr
    0 & 0 & \ldots & #2 \cr}}

% Produce a parenthesized row vector of variable #1 of length #2.
\def\rvec#1#2{\pmatrix{#1_{1} &\ldots & #1_{#2}}}

% Produce a parenthesized column vector of variable #1 of length #2.
\def\cvec#1#2{\pmatrix{#1_{1} \cr \vdots \cr #1_{#2}}}

% Produce a fraction #1/#2.
\def\frac#1#2{{{\scriptstyle #1} \over {\scriptstyle #2}}}

% Produce a fraction #1/#2 in display style.
\def\fracd#1#2{{{\displaystyle #1} \over {\displaystyle #2}}}

% Figure macro for eps files. #1 - eps file, #2 - caption text.
\count1=1
\def\epsfigure#1#2{%
  \bigskip
  \vbox{\hfil \epsfbox{#1} \hfil
      \vskip 0pt
      \hfil {Figure \number\count1: #2} \hfil
           \global\advance\count1 by 1
  \bigskip}}

% Insert the date.
\def\date{\monthname\ \number\day\ \number\year}

% Insert the date and time.
\def\datetime{\monthname\slash\number\day\slash\number\year \quad \timestring}

% My caligraphic initials.
\def\me{$\cal RSM$}

% My caligraphic initials and time stamp.
\def\mestamp{$\cal RSM$: \datetime}

% Item macro.
\def\item#1{%
{\leftskip = 4pc \parskip=0pt%
\par\noindent\llap{\hbox to 4pc{\hfil $\bf \bullet$ \hfil}}#1\par%
}%
}


% Begin enumeration marker.
\def\beginEnum{%
\count2=0
}

% Enumerate text.
\def\enum#1{%
  \global\advance\count2 by 1
  {\leftskip=2.5pc \parskip=0pt%
  \par\noindent\llap{\hbox to 2pc{\hfil $\bf \number\count2$. \hfil}}#1\par%
  }%
}

% End enumeration marker.
\def\endEnum{%
\count2=0
}
