\input macro

\parindent=0pt
\parskip=.5\baselineskip
\baselineskip = 1.1\baselineskip

\mantitle{Properties of a Linear Unbiased Minimum Variance Estimator}
{R. Scott McIntire}{Dec 29, 2023}

\section{Overview}
A standard problem is to find the linear unbiased ``minimum'' variance estimator
for the over determined problem\footnote{\kern 0.5pt \raise 0.5ex \hbox{\dag}}
{This means that $n > m$.}: $A {\bf x} = {\bf b}$, where 
$A$ is $nxm$ and ${\bf x}$ and
${\bf b}$ are $mx1$ and $nx1$ random variables respectively. The
covariance of ${\bf b}$ is given as $V$ and is assumed invertible. 
We look for a linear unbiased estimator for ${\bf x}$ of the form 
${\bf x}^* = L {\bf b}$, where $L\in {\bf R}^{mn}$. Since ${\bf x}^*$
is unbiased we must have 
$E{{\bf x} - {\bf x}^*} = E[{\bf x} - L
  {\bf b}] = E[{\bf x} - LA{\bf x}]$. We satisfy this constraint by
requiring that $LA = I$. This implies that $A$ and $L$ have rank $m$.
The covariance of $L{\bf b}$ is $LVL^T$. The problem now is to find
the best $L$ such that $LA = I$ and the covariance matrix, $LVL^T$, is
minimized in some sense.

The problem may be restated as choosing $L$ to ``minimize'' the norm of $LVL^T$
subject to the constraint $LA = I$. The question is what norm does one
use. The ``natural'' norm of a matrix is $\sqrt{{\rm
    tr}\left(M^TM\right)}$. This 
    follows from considering the standard basis in ${\bf R}^{nm}$ and
    using the usual norm of the vector space ${\bf R}^{nm}$. 
The matrix of interest, $LVL^T$ is positive definite
    symmetric; in this case, each of the ``trace'' 
expressions, $\root k \of {{\rm
          tr}\left(M^k\right)} \;\; (k \in [1,\infty ))$, is a norm for a 
positive definite matrix $M_{mxm}$.

\section{Do the Norms Matter?}
The question is, do different choices of norms lead to a different
solution, $L$? To answer this, we solve the constrained minimization 
problem that results from each of these trace norms\footnote{\kern 0.5pt *}
{Equivalently, we minimize the $k^{\rm th}$ power
of the $k^{\rm th}$ trace norm.}:
$$
F(L; k) = {\rm tr}\left((LVL^T)^k - (LA - I)\Lambda_{k}^T\right) 
\quad k \in [1,\infty)
$$
where, $\Lambda_k \in {\bf R}^{mm}$ is a Lagrange multiplier.

The derivative of $F$ with respect to $L$ is given by:
$$
\eqalignno{
DF(L)(H) & = {\rm tr}\left[ k(LVL^T)^{k-1}(HVL^T + LVH^T) - 
HA\Lambda_k^T\right]}
$$
At a minimum value, $L^*$, the derivative must vanish; therefore,
$$
\eqalignno{
& {\rm tr}\left[ k(L^*VL^{*T})^{k-1}(HVL^{*T} + L^*VH^T) - 
HA\Lambda_k^T\right] = 0
  \quad \forall H \in {\bf R}^{mn} & (1) \cr}
$$

Before going further we list some facts regarding the trace inner product that 
will be used in the rest of the paper.

\item{$\forall H \in R^{nn},\; H \; \hbox{ anti-symmetric} \; \Rightarrow \; {\rm tr}(H) = 0$.}
\item{Given a matrix $A \in R^{mn}$: $\forall H \in R^{mn},\; {\rm tr}(A^T H) = 0 \; \Rightarrow \; A \equiv 0.$}
\item{$\forall A,H \in R^{nn}, \;A \; {\rm symmetric} \; \wedge \; H \; \hbox{anti-symmetric}\;  \Rightarrow \; {\rm tr}(A^T H) = 0$.}
\item{Given a matrix $A \in R^{nn}$: $\forall H \in R^{nn}, \; H \; \hbox{anti-symmetric} \; \wedge \; {\rm tr}(A^T H) = 0 \; \Rightarrow A \; \hbox{is symmetric}$.}
\item{Given a matrix $A \in R^{nn}$: $\forall H \in R^{nn}, \; A \; \hbox{symmetric} \; \wedge \; H \; \hbox{symmetric} \; \wedge \; {\rm tr}(A^T H) = 0 \; \Rightarrow A \equiv 0$.}

The first fact follows from a property of square matrices; namely,
${\rm tr}(A) = {\rm tr}(A^T)$; and the definition of an anti-symmetric $A$, $A^T = -A$.
The second fact follows since any inner product on a vector space, $V$, satisfies: 
$\forall y \in V, \langle x, y \rangle = 0 \; \Rightarrow \; x \equiv 0$. 
The third fact follows since if $S$ and $W$ are symmetric and anti-symmetric matrices in $R^{nn}$, then
${\rm tr}(S \, W) = {\rm tr}((S \, W)^T) = {\rm tr}(W^T \, S^T) = -{\rm tr}(W \, S) = - {\rm tr}(S\, W)$
This implies that ${\rm tr}(S \, W) = 0$.
The fourth fact follows using that fact that any matrix $A \in R^{nn}$ can be written uniquely as a sum of a symmetric
and an anti-symmetric matrix: $A = A^{s} + A^{w}$. Therefore, ${\rm tr}(A^T H) = 0$ implies that
${\rm tr}((A^{s} - A^{w}) H)) = 0$. Using the second fact, ${\rm tr}((A^w)^T H) = 0$. Since this is true for any
anti-symmetric $H$, choosing $H$ to be $A^{w}$ and using the fact that the trace inner product induces a norm, we 
see that $A^{w} \equiv 0$. And thus $A$ is symmetric.
The fifth fact follows by setting $H$ to $A$. Again, the fact that the trace inner product induces a norm implies 
that $A = 0$.


Continuing with the derivation, in the special case $k = 1$, equation (1) 
implies that 
$ 2VL^{*T}  = A\Lambda_1^T$; or, 
$$
\eqalignno{
2L^{*T} & = V^{-1} A \Lambda_1^T & (2) \cr }
$$
Multiplying both sides by $A^T$ and using the fact that
$A^TL^{*T} = I$ yields $2I = A^TV^{-1}A\Lambda_1$. Since $A$ has rank
$m$, it follows that $\Lambda_1 = 2(A^TV^{-1}A)^{-1}$. Combining this
with equation (2) gives
$$
L^* = (A^TV^{-1}A)^{-1}A^TV^{-1}
$$

We now proceed with the case when $k > 1$. Given any ${\tilde H} \in {\bf R}^{mm}$, 
let $H = {\tilde H}A^TV^{-1}$. Then equation (1) becomes:

$$
\eqalignno{
& {\rm tr}\left[ k(L^*VL^{*T})^{k-1}({\tilde H}A^TL^{*T} + L^*A{\tilde H}^T) -
  {\tilde H}(A^TV^{-1}A)\Lambda_k^T\right]
= 0 \quad \forall {\tilde H} \in {\bf R}^{mm}  \cr }
$$

Or,
$$
\eqalignno{
& {\rm tr}\left[ k(L^*VL^{*T})^{k-1}({\tilde H} + {\tilde H}^T) -
  {\tilde H}(A^TV^{-1}A)\Lambda_k^T\right]
= 0 \quad \forall {\tilde H} \in {\bf R}^{mm} & (3)  \cr }
$$
since $L^*A = I$ and $A^TL^{*T} = I$.


Given any $H^\prime \in {\bf R}^{mm}$, consider ${\tilde H} =
H^{\prime}(A^TV^{-1}A)^{-1}$ with $H^\prime$
anti-symmetric. Since the trace of the product of symmetric 
and anti-symmetric square matrices is $0$, equation
(3) becomes ${\rm tr}(H^\prime\Lambda_k^T) = 0 \quad \forall
H_{mxm}^\prime$ with $H^\prime$ anti-symmetric. This implies that
$\Lambda_k$ is symmetric.

Equation (3) is true for all symmetric ${\tilde H}$. In this case it
reads:

$$
\eqalignno{
& {\rm tr}\left[ \biggl(2k(L^*VL^{*T})^{k-1} -
  \Lambda_k(A^TV^{-1}A)\biggr) {\tilde H}\right]
= 0 \quad \forall {\tilde H} \in {\bf R}^{mm} \;, \; {\tilde H} \; {\rm
  symmetric} & (3s)  \cr }
$$

Since $(L^*VL^{*T})^{k-1}$ and $\Lambda_k(A^TV^{-1}A)$ are both
symmetric, equation (3s) implies that 
$$
2k(L^*VL^{*T})^{k-1} - \Lambda_k(A^TV^{-1}A) = 0 
$$

Since $A$ has rank $m$, $(A^TV^{-1}A)^{-1}$ exists and we can solve for 
$\Lambda_k$.
$$
\Lambda_k = 2k(L^*VL^{*T})^{k-1}(A^TV^{-1}A)^{-1}
$$

Equation (1) now becomes:

$$
{\rm tr}\left[k(L^*VL^{*T})^{k-1}(HVL^{*T} + L^*VH^T) -
  2kHA(A^TV^{-1}A)^{-1}(L^*VL^{*T})^{k-1}\right] = 0 \quad 
\forall H \in {\bf R}^{mn}
$$

Or,

$$
{\rm tr}\left[\biggl( (HVL^{*T} + L^*VH^T) / 2 -
  HA(A^TV^{-1}A)^{-1}\biggr) (L^*VL^{*T})^{k-1}\right] = 0 \quad 
  \forall H \in {\bf R}^{mn}
$$

Which may be written
$$
{\rm tr}\left[\biggl( HVL^{*T} + ( L^*VH^T  - HVL^{*T} )/ 2 -
  HA(A^TV^{-1}A)^{-1}\biggr) (L^*VL^{*T})^{k-1}\right] = 0 \quad 
  \forall H \in {\bf R}^{mn}
$$

Since $( L^*VH^T  - HVL^{*T} )/ 2$ is anti-symmetric and $(L^*VL^{*T})^{k-1}$
is symmetric, the trace of their product is zero. Therefore we may write
the last equation as

$$
 {\rm tr}\left[\biggl( HVL^{*T} -
  HA(A^TV^{-1}A)^{-1}\biggr) (L^*VL^{*T})^{k-1}\right] = 0 \quad \forall H
    \in {\bf R}^{mn}
$$

Or,

$$
 {\rm tr}\left[H\biggl( VL^{*T} -
  A(A^TV^{-1}A)^{-1}\biggr) (L^*VL^{*T})^{k-1}\right] = 0 \quad \forall H
    \in {\bf R}^{mn} 
$$

This implies that $\left( VL^{*T} - A(A^TV^{-1}A)^{-1}\right)
(L^*VL^{*T})^{k-1} = 0 $.
Since $L^*$ has rank $m$, $(L^*VL^{*T})^{-(k-1)}$ exists, so that
$ VL^{*T} - A(A^TV^{-1}A)^{-1} = 0 $. Hence, 
$$
L^* = (A^TV^{-1}A)^{-1}A^TV^{-1}
$$


Therefore, it does not matter which of the matrix norms we
use: the "best" linear unbiased estimator is the same.

\vskip .25in
\section{Related features of ``the'' best estimator}
We now use the results of the trace norms above to show that 
the estimator obtained has the smallest maximum eigenvalue 
over all linear estimators.

\proclaim Lemma 1. If $\Omega$ is a closed set in $R^m$ and if 
$x^* \in \Omega$ is a solution to the family of
problems: \break ${\rm Min}\;\varphi_k(x)\;{\rm over}\;\{x \in \Omega:
g(x) = 0\} \; k \in [1,\infty)$; further, 
if  $\varphi_k$ are functions which converge point-wise to the
continuous function $\varphi$ on $\Omega$ with the accretive property
that 
$\lim_{\|x\| \rightarrow \infty}\limits
\varphi(x) \rightarrow \infty$, then 
$x^* \in \mathop{\rm Min}_{\{x \in \Omega: g(x) = 0\}}\limits \varphi(x) $

{\bf Proof:\/} 

{\it Existence:\/}
We know there is a non empty set of points that satisfy
$\{x \in \Omega: g(x) = 0\}$. If the minimum of $\varphi$ is attained 
on this set, then a solution exists. Otherwise, the infimum on this set 
is finite, by the accretive property of $\varphi$. 
Let $x_n$ be a sequence of points in this set 
such that $\varphi(x_n)$ converges to the infimum. Again, by the
accretive property, the $\{x_n\}$ are bounded in $\Omega$. Therefore,
since $\Omega$ is closed, 
there exists a subsequence of the $x_n$ which converges to a point, say
$x^\prime \in \Omega$. By the continuity of $\varphi$ on $\Omega$,
$x^\prime$ 
must attain the infimum of
$\varphi$ on the set $\{x \in \Omega : g(x) = 0\}$; so that a solution exists.

We proceed to show that $x^*$ is a solution.
Let ${\tilde x}$ be a solution of the problem 
${\rm Min}\; \varphi(x) \; {\rm over} \; \{x \in \Omega : g(x) = 0\}$;
then for any $\epsilon > 0$, we have, for sufficiently large $n$, 
(since $\varphi_n$ converges point-wise to $\varphi$)
$$
 \varphi_n({\tilde x}) < \varphi({\tilde x}) + \epsilon
$$
Since $x^*$ is a solution for $\varphi_n$, we also have
$$
\varphi_n(x^*) \le \varphi_n({\tilde x})
$$
Combining the two gives

$$
\varphi_n(x^*) \le \varphi_n({\tilde x}) < \varphi({\tilde x}) + \epsilon
$$

Or,
$$
\varphi_n(x^*) \le \varphi({\tilde x}) + \epsilon
$$

Taking the limit as $n \rightarrow \infty$ we have:

$$
\varphi(x^*) \le \varphi({\tilde x}) + \epsilon
$$

Since this is true for all $\epsilon > 0$ we necessarily have that
$$
\varphi(x^*) \le \varphi({\tilde x})
$$

Noting that $x^*$ satisfies the constraint, $g(x) = 0$, we infer that 
$x^*$ is also a solution.


\proclaim Theorem 1. The linear unbiased estimator previously obtained for the
least squares problem, $L^*{\bf b} =
(A^TV^{-1}A)^{-1}A^TV^{-1}{\bf b}$, is also the solution that
minimizes the maximum eigenvalue of the estimator's covariance matrix, 
$LVL^T$, subject to the constraint $L_{mxn}A_{nxm} = I_{mxm}$.

{\bf Proof:\/} We apply Lemma 1 with $\Omega$ equal to the $mxm$ symmetric
matrices; $\varphi$ equal to the $L_2$ matrix norm; and
$\varphi_n$ equal to the $n^{\rm th}$ trace norm. Theorem 1 follows
after collecting the facts: 

\item{The maximum eigenvalue of a symmetric matrix is the result of 
taking the limit of the trace norms of a symmetric matrix.}
\item{The $L_2$ norm applied to a symmetric matrix is the maximum
eigenvalue of that matrix.}
\item{The $L_2$ norm is accretive.}


\bye
