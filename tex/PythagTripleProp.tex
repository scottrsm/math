\documentclass[12pt]{article}

\usepackage{amsthm, amsmath, amssymb}
\usepackage{amssymb}
\usepackage{tikz}

\setlength{\textwidth}{6.5in}
\setlength{\textheight}{8.0in}
\setlength{\hoffset}{-0.75in}
\setlength{\voffset}{-0.75in}

\setlength{\parindent}{0.0in}
\setlength{\parskip}{1.5\baselineskip}

\title{An Alternative Proof of Elementary Properties of Pythagorean Triples}
\author{R. Scott McIntire}
\date{June 18, 2023}


\begin{document}

\maketitle

\section{Introduction}
When I was a freshmen in college I was briefly interested in the idea of 
using the areas of geometric figures to
produce interesting formulas. The example that inspired me was the way you 
can derive the Pythagorean formula
by examining the area of a square sliced into a few pieces.

I tried to do something similar with a more complicated geometric figure and 
I came up with figure $(\ref{fig_geo})$.
Using the fact that the radius of a circle is perpendicular to a 
tangent line, you can see that the larger triangle
is composed of 2 pairs of identical triangles and a square. 
The area of these triangles and the square must be
the same as the area of the larger triangle. 

\section{A Basic Property of Primitive Pythagorean Triples}
Writing the equivalence of the area of the figure and its parts we have:

\begin{eqnarray}
r^2 + 2 \frac{r(a - r)}{2} + 2 \frac{r(b-r)}{2} = \frac{a\, b}{2} 
\end{eqnarray}

Simplifying, this is:
\begin{eqnarray}
r^2 + r(a - r) + r(b-r) = \frac{a\, b}{2} \label{area} 
\end{eqnarray}

One can solve the quadratic for $r$ or read off from the figure 
that $(a-r) + (b-r) = c$. Solving this last equation for $r$ gives:
\begin{eqnarray}
  r & = & \frac{a + b - c}{2} \label{radius}
\end{eqnarray}

We now consider primitive Pythagorean triples (the length of sides from right 
triangles with integer sides and no common factors.)

What can we say about the even or oddness of the sides when the sides have no 
common factors? 

Since we know that $a^2 + b^2 = c^2$, we can reduce the combinations to 
the following:
\begin{itemize}
  \item{Both $a$ and $b$ are odd and $c$ is even.}
  \item{One and only one of either $a$ or $b$ is odd and $c$ is odd.}
\end{itemize}

Regardless of the which choice occurs, equation $(\ref{radius})$ implies that 
the radius of the circle is an integer.
Using this fact and equation $(\ref{area})$ we see that the quantity, 
$\frac{a \, b}{2}$, must be an integer. 
This knowledge eliminates the first choice; namely, that both $a$ and $b$ 
are odd. Consequently, one of the sides
is even. Let's say that it is $b$. 

Finally, regardless of whether $r$ is even or odd, the left hand side of 
$(\ref{area})$ is even. This implies that $\frac{a \, b}{2}$
is even. Again since, $a$ is odd, $b/2$ must be divisible by $2$. 

We have deduced the following properties for Primitive Pythagorean triples, 
$(a,b,c)$:
\begin{itemize}
  \item{One of the legs is odd; the other leg is even; and the hypotenuse 
      is odd.}
  \item{The even leg is divisible by $4$.}
\end{itemize}

\begin{figure}[!ht]
\label{fig_geo}
\begin{center}
  \begin{tikzpicture}[scale=0.75]
  % "a" leg.
  \draw (0,0)  --  (15,0) ;

  % "b" leg.
  \draw (15,0) --  (15,20);

  % "c" leg.
  \draw (0,0)  --  (15,20);

  % The circle and its three "r" spokes.
  \draw (10,5)  circle (5);
  \draw (10,5) --  (10,0) ;
  \draw (10,5) --  (15,5) ;
  \draw (10,5) --  (6,8)  ;

  % Label the "r" spoke pointing at the hypotenusen{ce.
  \node at (8,5.5) {\Large $r$};

  % Label the hypotenuse.
  \node at (5,9) {\Huge $c$};
  % Label the hypotenuse.

  % Label the right "leg".
  \node at (16,9) {\Huge $b$};

  % Label the bottom "leg".
  \node at (8,-1) {\Huge $a$};

  % Label the legs -- less the side of the square.
  \node at (17,12) {\Large $(b - r)$};
  \node at (4,-1) {\Large $(a - r)$} ;

  % Label the edges of the square.
  \node at (12,-1) {\Large $r$};
  \node at (16,2) {\Large $r$} ;

  % Draw lines from the left and top vertices to the center of the circle.
  \draw (10,5) -- (0,0)  ;
  \draw (10,5) -- (15,20);

  % Label the legs.
  \node at (7,12) {\Large $(b - r)$};
  \node at (0,3) {\Large $(a - r)$} ;

\end{tikzpicture}
\end{center}
\caption{Dissected Triangle Figure}
\end{figure}

\end{document}

