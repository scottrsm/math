\documentclass{article}

\usepackage{amsthm, amsmath, amssymb}
\usepackage{amssymb}

\setlength{\textwidth}{6.5in}
\setlength{\textheight}{8.0in}
\setlength{\hoffset}{-0.75in}
\setlength{\voffset}{-0.75in}

\setlength{\parindent}{0.0in}
\setlength{\parskip}{1.5\baselineskip}

\title{Conjectures for Primitive Pythagorean Triples}
\author{R. Scott McIntire}
\date{July 11, 2023}

\newtheorem{theorem}{Theorem}\theoremstyle{plain}
\newtheorem{thm}{Theorem}

\theoremstyle{definition}
\newtheorem{defn}[thm]{Definition}

\newtheorem{conjecture}{Conjecture}\theoremstyle{plain}


\begin{document}
\maketitle


\section{Overview}
We list a few conjectures regarding the set, ${\cal H}$, of primitive 
Pythagorean hypotenuses. This is the set constructed by taking the 
``hypotenuse'' value from all primitive Pythagorean triples -- corresponding 
to the lengths of the sides of right triangles with integer sides with 
no common factors.

A related set, ${\cal H}_p$, is the subset of ${\cal H}$ that are primes. 
That is, primitive Pythagorean hypotenuses that are prime.

Below we list the main conjectures informally, listing them in order of 
complexity:
\begin{enumerate}
  \item{{\bf Partition of ${\cal H}$:\/} ${\cal H}$ can be {\em partitioned\/} 
      into two disjoint infinite sets: 
    the set of all powers of Pythagorean primes; and the set of all 
        hypotenuses that come from more than one primitive Pythagorean triple.}
  \item{{\bf ${\cal H}_p$ is complex:\/} The set of Pythagorean primes, 
      ${\cal H}_p$, is infinite; {\em similar\/}; and as {\em complex\/} as 
        the primes when measured using certain metrics.}
\end{enumerate}


\section{Definitions}
After a few definitions we describe a conjecture regarding Primitive 
Pythagorean Triples. We use standard notation to indicate that the natural 
numbers are represented by the symbol $\mathbb{N}$ and 
we use the symbol $\mathbb{P}$ to denote the set of prime numbers.


\defn{\em Pythagorean Triples\/} denoted, ${\cal T}$, are  triplets of 
numbers representing the length of the sides of
right triangles where all the sides are integers. Specifically,
\begin{eqnarray}
  {\cal T} & = & \{ (x,y,z) \; | \; (x,y,z) \in \mathbb{N}^3 \; \wedge \; x^2 + y^2 = z^2 \}
\end{eqnarray}
The usual definition.

\defn{\em Primitive Pythagorean Triples\/} denoted, ${\cal T}_p$, are 
Pythagorean Triples which have no common factors. Specifically,
\begin{eqnarray}
  {\cal T}_p & = & \{ (x,y,z) \; | \; (x,y,z) \in \mathbb{N}^3 \; \wedge \; \gcd(x,y) = 1 \; \wedge \; x^2 + y^2 = z^2 \}
\end{eqnarray}
The usual definition.

\defn{\em Primitive Pythagorean Hypotenuses\/} denoted, ${\cal H}$, defined by:
\begin{eqnarray}
  {\cal H} & = & \{ z \; | \; \exists (x,y) \in \mathbb{N}^2, \;  (x,y,z) \in {\cal T}_p \}
\end{eqnarray}
This represents the set of all possible hypotenuses of primitive Pythagorean triples.

\defn{\em Pythagorean Primes\/} denoted, ${\cal H}_p$, defined by:
\begin{eqnarray}
  {\cal H}_p & = & \{ z \; | \; \exists (x,y) \in \mathbb{N}^2, \, (x,y,z) \in {\cal T}_p \; \wedge \; z \in \mathbb{P} \}
\end{eqnarray}
This represents the set of all possible lengths of hypotenuses of right 
triangles with integer sides that are prime.

\defn{\em Duplicate Primitive Hypotenuses\/} denoted ${\cal H}_d$, defined by:
\begin{eqnarray}
  {\cal H}_{d} & = & \{ z \; | \; \exists\, (x_1, x_2, y_1, y_2) \in \mathbb{N}^4, \; 
      (x_1,y_1,z) \in {\cal T}_p \; \wedge \; (x_2, y_2, z) \in {\cal T}_p \; \wedge \; x_1 \neq x_2 \}  
\end{eqnarray}
This represents the set of all primitive hypotenuses that can be found in 
more than one primitive Pythagorean triple.

\defn{\em  Primitive Hypotenuses Powers\/} denoted ${\cal H}_u$, defined by:
\begin{eqnarray}
  {\cal H}_u & = & \{ z \; | \; \exists\,(p,n) \in ({\cal H}_p, \mathbb{N}), \; z = p^n \}
\end{eqnarray}
This represents the set of all powers of Pythagorean primes.


\defn{\em The Prime Parity\/} function, $\Pi$, is defined on any prime, $p$, with $p \ge 5$ by:
\begin{eqnarray}
  \Pi(p) = \begin{cases}
  \hphantom{-}1\hfil & \exists n \in \mathbb{N}, p = 6n + 1 \\
            -1 \hfil & \exists n \in \mathbb{N}, p = 6n - 1 \\
             \end{cases} 
\end{eqnarray}

\defn{\em First Deviation\/} function, $\Delta_p$, is defined by:%
\footnote{If for a given $n$, 
$\left\{ m \; \left| \; n \le \left( \sum_{i=1}^m \Pi({\cal H}_p[i]) - \sum_{i=3}^{m+2} \Pi(\mathbb{P}[i]) \right. \right) \right\} = \emptyset$, we set $\Delta_p(n) = \infty$.}
\begin{eqnarray}
\Delta_p(n) = \mathop{\rm min}\limits_{m} \left\{ m \; \left| \; n \le \left( \sum_{i=1}^m \Pi({\cal H}_p[i]) - \sum_{i=3}^{m+2} \Pi(\mathbb{P}[i])\right. \right) \right\} 
\end{eqnarray}

{\bf Notes:} 
\begin{itemize}
  \item{The subscript $u$ in the name ${\cal H}_u$ is not yet justified.}
  \item{${\cal H}_p$ is a {\em proper\/} subset of the primes: ${\cal H}_p \subset \mathbb{P}$.}
  \item{If the sets above are treated as lists, they are indexed in sorted order.}
  \item{Clearly, $\Delta_p$ is a non-decreasing function.}
\end{itemize}

First we list a few facts and state a couple of known theorems before proceeding:
\begin{enumerate}
  \item{The Pythagorean primes are of the form $4n + 1$. 
      This is a special case of a more general theorem proven by 
        Pierre de Fermat, known as Fermat's theorem on sums of two squares. 
      According to this theorem, an odd prime number $p$ can be expressed as 
        the sum of two squares if and only if 
      $p$ is congruent to $1$ modulo $4$, i.e., $p = 4n + 1$ for some 
        positive integer $n$. 
    \item{Moreover, the distribution of primes of the form $4n + 1$ has 
        been studied extensively.}
      The German mathematician Peter Gustav Lejeune Dirichlet proved that 
        there are infinitely many primes of the form $4n + 1$, a result known as 
      Dirichlet's theorem on arithmetic progressions. 
      This theorem, which generalizes the case of Pythagorean primes, states 
        that for any two positive coprime numbers a and d, 
      there are infinitely many primes of the form $a + nd$, where $n$ is a 
        non-negative integer.} 
    \item{In the case of Pythagorean primes, we can take a = 1 and d = 4 to 
        get the form 4n + 1. 
          Dirichlet's theorem thus guarantees that there are infinitely 
          many Pythagorean primes. 
        This is a profound result, as it extends the prime number theorem, 
        which states that there are infinitely many prime numbers, to 
        arithmetic progressions.}
\end{enumerate}
From these results we get the following structure theorem for 
    Pythagorean hypotenuses:
\theorem{The set ${\cal H}_p = \{ p \; | \; p \in \mathbb{P} \; \wedge \; p = 1 \; {\rm mod}\;  4\}$}
\theorem{The set ${\cal H}$ has a unique factorization in ${\cal H}_p$. }
         That is, each element of ${\cal H}$ can be uniquely written as a product of powers from ${\cal H}_p$:
         \begin{eqnarray}
           \forall h \in {\cal H},\, \exists ! N \in \mathbb{N}, \, \exists ! ({\bf p},{\bf k}) \in \left({\cal H}_p^{N}, \mathbb{N}^{N}\right): h = \prod_{i=1}^N p_{i}^{k_i}.
         \end{eqnarray}


\section{McIntire's Pythagorean Triple Conjectures}

\conjecture{The set ${\cal H}$ is partitioned by the sets ${\cal H}_d$ and ${\cal H}_u$:}
\begin{enumerate}
  \item{$\hphantom{|{\cal H}_d|}{\cal H} = {\cal H}_{d} \cup {\cal H}_u$;}
  \item{$\hphantom{|{\cal H}_d|}\emptyset = {\cal H}_{d} \cap {\cal H}_u$};
  \item{$\hphantom{\emptyset}|{\cal H}_d| = \infty$.}
\end{enumerate}

In simple terms the conjecture says that each element of the set of all 
primitive Pythagorean hypotenuses is the union of two disjoint infinite sets:
the set of powers of a Pythagorean primes; and the set of all hypotenuses 
that come from more than one primitive Pythagorean triple.

\conjecture{Properties of the cumulative parity of ${\cal H}_p$ indicates 
that it is non-trivial:}
\begin{enumerate}
  \item{$\lim\limits_{n \rightarrow \infty} \frac{\sum\limits_{i=1}^n \Pi({\cal H}_p[i])}{n} = 0$;}
  \item{$\forall n\in \mathbb{N}, \Delta_p(n) < \infty$;}
  \item{$ \lim\limits_{n \rightarrow \infty} \frac{\Delta_p(n)}{n^{\ln(2 \pi)}} = 41$.}
\end{enumerate}

These last conjectures suggest that the Pythagorean primes are non-trivial 
in a way similar to the primes.
\begin{enumerate}
  \item{The first conjecture states that the ${\cal H}_p$ primes are 
      "spread out" in a way similar to the set of all primes.}
  \item{From numerical experiments, its seems that the cumulative parity 
      of ${\cal H}_p$ is "on average" lower than the cumulative parity of all primes.
      The second and third conjectures suggest that there 
      are non-trivial deviations from this rule. This "mixing" of the 
        cumulative parity of ${\cal H}_p$ with that of $\mathbb{P}$ 
      suggests that the Pythagorean primes are as "complex" as the set of 
        all primes.}
\end{enumerate}

\end{document}


